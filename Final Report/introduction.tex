\section{Introduction}
\par This project aims to study the feasibility of formalising
Automata Theory \cite{aho1972} in Type Theory
\cite{martin1984} with the aid of a dependently-typed functional
programming language, Agda. Automata Theory is an extensive work;
therefore, it will be unrealistic to
include all the materials under the time constraints. Accordingly,
this project will only focus on the theorems and
proofs that are related to the translation of regular expressions
to finite automata. In addition, this project also serves as an
example of how complex and non-trivial proofs are formalised. 

\par Our Agda formalisation consists of two components: 1) the
translation of regular expressions to a minimal DFA and 2)
the correctness proofs of the translation. At this stage, we are only
interested in the correctness of the translation but not the
efficiency of the algorithms. 


\subsection{Background}
\par Type theory was introduced by Per Martin-L{\"o}f in
1971 to provide an alternative foundation of mathematics based on the
principles of mathematical constructivism where logic can be
implemented using the theory. From another perspective, Type Theory is also a
dependently-typed functional programming language. In order to bridge the gap between
the theoretical representation of type theory and the requirements on
a practical programming language, Norell \cite{norell2007} rewrote a
dependently-typed language, Agda, based on Type Theory. Agda allows us
to formalise mathematics and programming logic in Type Theory and to 
check the formalisation automatically by using its type checker and termination
checker. We will discuss the use of Agda in Section
2 in details. 
\par Automata Thoery is the study of abstract machines and
automata. Automata are abstract models of machines that perform
computations on an input by moving between its states. There
are several major families of automaton but we will
only focus on finite state machine or finite state automata. 


\subsection{Motivation}
\par My motivation on this project is to learn and apply
dependent types in formalising programming logic. At the beginning, I
was new to dependent types and proof assistants; therefore, we
had to choose carefully what theorems to formalise. On one hand, the theorems
should be non-trivial enough such that a substantial amount of work is required
to be done. On the other hand, the theorems should not be too
difficult because I am only a beginner in this area. Finally, we
decided to go with the Automata Theory as its basic concepts were
explained in the course \textit{Model of Computation}. 


\subsection{Outline}
\par Section 2 will be a brief introduction on Agda and
dependent types. We will describe how Agda can be used as a proof
assistant by giving examples of formalised proofs. Experienced Agda
users can skip this section and start from section 3 directly. In
section 3, we will describe several researches
that are also related to the formalisation of Automata
Theory. Following the background, section 4 will be a detail description of our
work. We will walk through the two components of our Agda
formalisation. Note that the definitions,
theorems and proofs written in this section are extracted from our
Agda code. They may be different from
their usual mathematical forms in order to adapt to the environment of
Type Theory. In section 5, we will discuss two possible extensions
to our project: 1) Myhill-Nerode Theorem and 2) the Pumping
Lemma. After that, in section 6, we will evaluate the project as a
whole. Finally, the conclusions will be drawn. 