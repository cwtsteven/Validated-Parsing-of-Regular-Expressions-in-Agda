\section{Introduction}
\paragraph{} This project aims to study the feasibility of formalising
Automata Theory \cite{aho1972} in Type Theory with the aid of a dependently-typed
functional programming language, Agda \cite{agdawiki2016}. Automata
Theory is an extensive work; therefore, it will be unrealistic to
include all the materials under the time constraint. Accordingly, this project will only focus on the theorems and
proofs that are related to the translation between regular expressions
and finite automata. In addition, this project also gives a brief introduction
on how complex and non-trival proofs are formalised. 

\paragraph{} Our Agda formalisation is consist of two major
components: 1) the translation of regular expressions to DFA and 2)
the correctness proofs of the translation. At this stage, we are only
interested in the correctness of the translation but not the
efficiency of the algorithms. 


\subsection{Motivation}
\paragraph{} My motivation on this project is to learn and apply
dependent types in programming as well as in formalising programming logic in
Agda. ... 


\subsection{Overview}
\paragraph{} In the beginning of section two, we will give a brief
introduction on Agda and dependent types. In addition, we will also look into some small Agda proofs so that readers can
have a taste of how proofs are formalised in Type Theory. In the end
of section two, we will describe a similar reseach
conducted by Firsov and Uustalu \cite{firsov2013}. 

\paragraph{} Following the
background, the third section will be a detail description of our
work. In this section, we will walk through the Agda
formalisation of the two major components. All the definitions,
theorems and proofs written in this report are extracted from our Agda
code to adapt the environment in Type Theory. 

\paragraph{} In the fourth section, we will discuss two possible extensions
to the project: 1) Myhill-Nerode Theorem and 2) the Pumping
Lemma. After that, we will evaluate the project as a
whole. Finally, the conclusions will be drawn. 