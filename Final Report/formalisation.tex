\theoremstyle{definition}
\newtheorem{defn}{Definition}
\theoremstyle{plain}
\newtheorem{thm}{Theorem}
\theoremstyle{plain}
\newtheorem{lem}{Lemma}

\def\defnautorefname~#1\null{%
  Definition~(#1)\null
}
\def\thmautorefname~#1\null{%
  Theorem~(#1)\null
}
\def\lemautorefname~#1\null{%
  Lemma~(#1)\null
}

\chapter{Formalisation in Type Theory}
\par Let us recall the two components of our formalisation: the
translation of regular expressions to a minimal DFA and the correctness proofs
of the translation. The translation is
divided into several steps. Firstly, a regular expression is
converted into an \(\epsilon\)-NFA using Thompson's construction
\cite{thompson1968}. Secondly, all the \(\epsilon\)-transitions are
removed by computing the \(\epsilon\)-closures. Thirdly, a DFA is built by using powerset
construction. After that, all the unreachable
states are removed. Finally, a MDFA is obtained by using
quotient construction. The translation is correct if and only if 1) the
accepting languages of the regular expression and its translated
output are equal, i.e. \(L(regex) = L(translated\) \(\epsilon\)-NFA\() = L(translated\) DFA\() =
L(translated\) MDFA\()\) and 2) the translated MDFA is minimal. 

\par In this chapter, we will walk through the formalisation of
each of the above steps together with their correctness proofs. Note
that all the definitions, theorems, lemmas and proofs written in this section
are adapted to the environment of Agda. Now, let us begin with the 
representation of subsets. 

\subsection{Subsets, Decidable Subsets and Vector Representation}
\par The types of subsets and decidable subsets are defined in
\textbf{Subset.agda} and \textbf{Subset/DecidableSubset.agda}
respectively along with their operations such as membership (\(\in\)), subset
(\(\subseteq\)), superset (\(\supseteq\)) and equality (\(=\)). Let us
begin with the definition of general subsets. To separate the
operations of subsets and decidable subsets, all the operations of
decidable subset are denoted by the superscript (\(^d\)), e.g. \(\in^d\)
is the membership decider for decidable subsets. 

\begin{defn} 
\noindent Suppose \(A\) is a set, then its
subsets are represented as unary functions on
\(A\) in Type Theory, i.e. \(Subset\ A = A \to Set\). 
\end{defn}

\par In our definition, a subset is a function from \(A\) to
\(Set\). When declaring a subset, we can write \(sub =
\lambda\ (x : A) \to P\ x\). \(P\ x\) defines the conditions for \(x\) to
be included in \(sub\). This construction is
very similar to set comprehension. For example, the above subset
resembles the set \(\{x\ | \ x \in A,\ P(x)\}\). Furthermore, \(sub\) is
also a predicate on \(A\) as its type is in the form of \(A \to
Set\) and its decidability will remain unknown until it is either proved or disproved. 

\begin{defn} 
\noindent Another representation of subsets is \(DecSubset\ A = A \to
Bool\). Unlike \(Subset\), its decidability is ensured by its
definition. 
\end{defn}

\par The two representations have different roles in the project. For
example, \(Language\) is defined using \(Subset\) as not every
language is decidable. For other parts in the project 
such as the subsets of states in an automaton, \(DecSubset\) is used
because the decidability is assumed. 


\subsubsection{Vector Representation}
\par At the beginning, we intended to avoid the vector representation of a
subset. However, it is impossible as we have to iterate the subset
when computing \(\epsilon\)-closures. The problem will be discussed in
Section 6 in details. The vector representation is defined in
\textbf{Subset/VectorRep.agda} along with its operations and proofs. 

\par The most important part is in the module \textbf{Vec-Rep} where
several objects must be established before in order to use the vector
representation. They are \((A : Set)\) -- a set with finite elements;
\((dec : DecEq A)\) -- the decidable equality of \(A\); \((n : \mathbb
N)\) -- number of elements minus 1; \((It : Vec\ A\ (suc\ n))\) -- a
vector containing elements of \(A\) with length \(suc\ n\); \((\forall
a\in It)\) -- a proof that any state in \(A\) is also in the vector
\(It\); and \((unique : Unique)\) -- a proof that there is no repeating
element in \(It\). 

\par Apart from iterating a subset, the vector representation allows
us to prove the following lemmas. 

\begin{defn}
\noindent We define \(any\) as a predicate of vector such that it is
true if and only if there exists an element in the vector that satisfies a given
predicate \(P\). 
\end{defn}

\par It is defined in Agda as follow:
\begin{lstlisting}[mathescape=true,xleftmargin=.3\textwidth]
any : {A : Set}{n : $\mathbb N$}(P : A $\to$ Set) $\to$ Vec A n $\to$ Set
any P []        = $\bot$
any P (a :: as) = P a $\uplus$ any P as
\end{lstlisting} 

\begin{lem}
\noindent For a set \(A\) and any proposition \(P\), there exists an
element in \(It\) that satisfies \(P\) if and only if there exists an
element in \(A\) that satisfies \(P\). 
\end{lem}

\begin{proof}
\noindent The proof looks quite obvious. Since \(It\) contains all the
elements of \(A\), the statement must be true. However, in Type
Theory, we have to prove it by induction on the vector.
\end{proof}

\begin{defn}
\noindent We define \(all\) as a predicate of vector such that it is
true if and only if all the elements in the vector satisfy a given
predicate \(P\). 
\end{defn}

\par It is defined in Agda as follow:
\begin{lstlisting}[mathescape=true,xleftmargin=.3\textwidth]
all : {A : Set}{n : $\mathbb N$}(P : A $\to$ Set) $\to$ Vec A n $\to$ Set
all P []        = $\top$
all P (a :: as) = P a $\times$ all P as
\end{lstlisting} 

\begin{lem}
\noindent For a set \(A\) and any proposition \(P\), all
the elements in \(It\) that satisfy \(P\) if and only if all the 
elements in \(A\) satisfy \(P\). 
\end{lem}

\begin{proof}
\noindent Again, the proof looks quite obvious. Since \(It\) contains all the
elements of \(A\), the statement must be true. However, in Type
Theory, we have to prove it by induction on the vector.
\end{proof}

\par Apart from the above lemmas, the vector representation also allows
us to get some useful proofs when we are arguing the size of a
vector. 


\subsection{Languages}
\par The type of languages is defined in \textbf{Language.agda} along with its 
operations and lemmas such as union (\(\cup\)), concatenation
(\(\bullet\)) and closure (\(\star\)). 

\par We represent the set of alphabets \(\Sigma\) as a data type in
Type Theory, i.e. \(\Sigma : Set\). Note that the equality relation of \(\Sigma\) needs to be
decidable. In Agda, they are passed to every module as
parameters, e.g. \(module\ Language (\Sigma : Set)\ (dec : DecEq\
\Sigma)\ where\). 

\begin{defn}
\noindent We first define \(\Sigma^*\) as the set of all
strings over \(\Sigma\). In our approach, it is expressed as a list of
alphabets, i.e. \(\Sigma^* = List\ \Sigma\). 
\end{defn}

\par For example, (\(A :: g :: d :: a :: [\ ]\)) is equivalent to the
string 'Agda' and the empty list \([\ ]\)
represents the empty string (\(\epsilon\)). In this way, the first
alphabet can be extracted from the input string by pattern matching in order to
run a transition from a particular state to another state in an automaton. 

\begin{defn}
\noindent A language is defined as a subset of 
\(\Sigma^*\), i.e. \(Language = Subset\ \Sigma^*\). 
Note that \(Subset\) instead of \(DecSubset\) is used because not
every language is decidable. 
\end{defn}

\subsubsection{Operations on Languages}

\begin{defn} 
\label{defn:lang_union}
\noindent Suppose \(L_1\) and \(L_2\) are languages, then the union of
the two languages, \(L_1\cup L_2\), is given by the set \(\{w\
|\  w \in L_1\ \vee \ w \in L_2\}\). In Type Theory, we have \(L_1 \cup L_2 = \lambda\ w
\to w \in L_1\ \uplus\ w \in L_2\).
\end{defn}

\begin{defn}
\label{defn:lang_con}
\noindent Suppose \(L_1\) and \(L_2\) are languages, then
the concatenation of the two languages, \(L_1\bullet L_2\), is given
by the set \(\{w\  |\  \exists u\in L_1.\ \exists v\in L_2.\ w = uv\}\). In
Type Theory, we have \(L_1\bullet L_2 = \lambda\ w \to \exists[\
u \in \Sigma^*\ ]\ \exists[\ v \in \Sigma^*\ ] (u \in L_1 \times v \in
L_2 \times w \equiv u\ \Plus\Plus\  v )\).
\end{defn}

\begin{defn}
\label{defn:lang_power}
\noindent Suppose \(L\) is a language, then we define \(L^n\) as
the concatenation of \(L\) with itself over \(n\) times, i.e. \(L^n =
L\bullet L\bullet L ... L\). In Type Theory, it is defined as a recursive function where \(L \wedge zero = [\![\epsilon ]\!]\) and
\(L \wedge (suc\ n) = L \bullet (L \wedge n)\). 
\end{defn}

\begin{defn}
\label{defn:lang_star}
\noindent Suppose \(L\) is a language, then the closure of
L, \(L\ast\) is given by the set \(\bigcup_{n \in N} L^n\). In Type
Theory, we have \(L\ \star = \lambda\ w \to \exists [\ n \in \mathbb{N}\
]\ (w \in L \wedge n)\). 
\end{defn}


\subsection{Regular Expressions and Regular Languages}
\par The types of regular expressions and regular languages are defined in
\textbf{RegularExpression.agda}. 

\begin{defn}
\label{defn:regex}
\noindent Regular expressions over \(\Sigma\) are defined inductively as follow: 
\begin{enumerate}[nolistsep]
  \item \(\O\) is a regular expression denoting the regular language \(\O\);
  \item \(\epsilon\) is a regular expression denoting the regular language \(\{\epsilon\}\);
  \item \(\forall a\in\Sigma.\ a\) is a regular expression denoting the regular language \(\{a\}\);
  \item if \(e_{1}\) and \(e_{2}\) are regular expressions denoting the regular
    languages \(L_1\) and \(L_2\) respectively, then
    \begin{enumerate}[nolistsep]
      \item \(e_{1}\ |\ e_{2}\) is a regular expressions denoting the
        regular language \(L_1 \cup L_2\);
      \item \(e_{1}\cdot e_{2}\) is a regular expression denoting the
        regular language \(L_1\bullet L_2\);
      \item \(e_{1}^{\ *}\) is a regular expression denoting the regular
        language \(L_1\ \star\).
     \end{enumerate}
\end{enumerate}
\end{defn}

\par The interpretation of regular expression in Agda is as follow:

\begin{lstlisting}[mathescape=true,xleftmargin=.3\textwidth]
data RegExp : Set where
  $\O$    : RegExp
  $\epsilon$    : RegExp
  $\sigma$    : $\Sigma$ $\to$ RegExp
  _|_ : RegExp $\to$ RegExp $\to$ RegExp
  _$\cdot$_  : RegExp $\to$ RegExp $\to$ RegExp
  _$^*$  : RegExp $\to$ RegExp
\end{lstlisting} 

\par The accepting language of regular expressions is defined as
a function from \(RegExp\) to \(Language\). 

\begin{lstlisting}[mathescape=true,xleftmargin=.3\textwidth]
L$^R$ : RegExp $\to$ Language
L$^R$ $\O$   = $\o$
L$^R$ $\epsilon$   = $[\![\epsilon ]\!]$
L$^R$ ($\sigma$ a) = $[\![\ a\ ]\!]$
L$^R$ (e$_1$ | e$_2$) = L$^R$ e$_1$ $\cup$ L$^R$ e$_2$
L$^R$ (e$_1$ $\cdot$ e$_2$) = L$^R$ e$_1$ $\bullet$ L$^R$ e$_2$
L$^R$ (e$^*$) = (L$^R$ e) $\star$
\end{lstlisting} 

\section{\(\epsilon\)-Non-deterministic Finite Automata}
\par Recall that the set of all strings over \(\Sigma\) is defined as
the type \(List\ \Sigma^*\). However, this definition gives us no way to
extract an \(\epsilon\) alphabet from the input string. Therefore,
we need to introduce another representation specific to this
purpose. The representation in defined under the module
\textbf{\(\Sigma\)-with-\(\epsilon\)} in \textbf{Language.agda} along
with itis related operations and lemmas. 

\begin{defn}
\noindent We define \(\Sigma^e\) as the union of
\(\Sigma\) and \(\{\epsilon\}\), i.e. \(\Sigma^e = \Sigma \cup \{\epsilon\}\).
\end{defn} 

\par The equivalent data type is as follow:
\begin{lstlisting}[mathescape=true,xleftmargin=.3\textwidth]
data $\Sigma^e$ : Set where
  $\alpha$ : $\Sigma \to \Sigma^e$
  E : $\Sigma^e$
\end{lstlisting}

\par All the alphabets in \(\Sigma\) are included in \(\Sigma^e\) by using the
\(\alpha\) constructor while the \(\epsilon\) alphabet corresponds to
the constructor \(E\) in the data type. 

\begin{defn}
\noindent Now we define \(\Sigma^{e*}\), the set of all strings over
\(\Sigma^e\) in a way similar to \(\Sigma^*\), i.e. \(\Sigma^{e*} =
List\ \Sigma^e\). 
\end{defn}

\par For example, the string 'Agda' can be
represented by (\(\alpha\ A :: \alpha\ g :: E :: \alpha\ d :: E :: \alpha\
a :: [\ ]\)) or (\(E :: \alpha\ A :: E :: E :: \alpha\ g :: \alpha\ d ::
E :: \alpha\ a :: [\ ]\)). We call these two lists as the \(\epsilon\)-strings of the
string 'Agda'. 

\begin{defn}
\noindent Now we define \(to\Sigma^*(w^e)\) as a function that takes an
\(\epsilon\)-string of \(w\), \(w^e\) and returns \(w\). 
\end{defn}

\par It is define in Agda as follow: 
\begin{lstlisting}[mathescape=true,xleftmargin=.3\textwidth]
to$\Sigma^*$ : $\Sigma^{e*}$ $\to$ $\Sigma^*$
to$\Sigma^*$ [] = []
to$\Sigma^*$ ($\alpha$ a :: w) = a :: to$\Sigma^*$ w
to$\Sigma^*$ (E :: w) = to$\Sigma^*$ w
\end{lstlisting}

\par Now, let us define \(\epsilon\)-NFA using \(\Sigma^{e*}\). The
type of \(\epsilon\)-NFA is defined in \textbf{eNFA.agda} along with its operations and
properties. 

\begin{defn}
\noindent An \(\epsilon\)-NFA is a 5-tuple \(M = (Q
,\ \Sigma^e,\ \delta,\ q_0,\ F)\), where
\begin{enumerate}[nolistsep]
  \item \(Q\) is a finite set of states;
  \item \(\Sigma^e\) is the union of \(\Sigma\) and \(\{\epsilon\}\);
  \item \(\delta\) is a mapping from \(Q \times \Sigma^e\) to
    \(\mathcal P \left({Q}\right)\) that defines the behaviour of the automata;
  \item \(q_0\) in \(Q\) is the initial state;
  \item \(F \subseteq Q\) is the set of accepting states. 
\end{enumerate}
\end{defn}

\par It is formalised as a record in Agda as shown below: 

\begin{lstlisting}[mathescape=true,xleftmargin=.3\textwidth]
record $\epsilon \hyphen$NFA : Set$_1$ where
  field
    Q      : Set
    $\delta$       : Q $\to$ $\Sigma^e$ $\to$ DecSubset Q
    q$_0$      : Q
    F      : DecSubset Q
    $\forall$qEq    : $\forall$ q $\to$ q $\in^d$ $\delta$ q E
    Q?     : DecEq Q
    |Q|-1  : $\mathbb{N}$
    It     : Vec Q (suc |Q|-1)
    $\forall$q$\in$It    : (q : Q) $\to$ (q $\in^V$ It)
    unique : Unique It
\end{lstlisting}

\par The set of alphabets \(\Sigma\) is passed into the module as a
parameter and \(\Sigma^e\) is constructed using \(\Sigma\). Together with \(Q\), \(\delta\),
\(q_0\) and \(F\), these five fields correspond to the 5-tuple
\(\epsilon\)-NFA. The other extra fields are used when computing
\(\epsilon\)-closures. They are \(\forall qEq\) -- a proof that any
state in \(Q\) can reach itself by an
\(\epsilon\)-transition; \(Q?\) -- the decidable equality of \(Q\);
\(|Q|-1\) -- the number of states minus 1; \(It\) -- a vector
containing all the states in \(Q\); \(\forall q\in It\)
-- a proof that every state in \(Q\) is also in the vector
\(It\); and \(unique\) -- a proof that there is no repeating elements in
\(It\). 

\par Now, before we can define the accepting language of a given
\(\epsilon\)-NFA, we need to define several operations of
\(\epsilon\)-NFA. 

\begin{defn}
\noindent A configuration is composed of a state and an alphabet from
\(\Sigma^e\), i.e. \(C = Q \times \Sigma^e\). 
\end{defn}

\begin{defn}
\noindent A move in an \(\epsilon\)-NFA is
represented by a binary function (\(\vdash\)) on two configurations. We say
that for all \(w \in \Sigma^{e*}\) and \(a \in \Sigma^e\), \((q, aw)
\vdash (q' , w)\) if and only if \(q' \in \delta (q , a)\). 
\end{defn}

\par The binary function is defined in Agda as follow: 
\begin{lstlisting}[mathescape=true,xleftmargin=.3\textwidth]
  _$\vdash$_ : (Q $\times$ $\Sigma^e$ $\times$ $\Sigma^{e*}$) $\to$ (Q $\times$ $\Sigma^{e*}$) $\to$ Set
  (q , a , w) $\vdash$ (q' , w') = w $\equiv$ w' $\times$ q' $\in^d$ $\delta$ q a
\end{lstlisting}

\begin{defn}
\noindent Suppose \(C\) and \(C'\) are two configurations. We say that \(C \vdash^0 C'\) if and only
if \(C = C'\); and \(C_0 \vdash^k C_k\) for any \(k \geq 1\) if and only if there exists a chain of
configurations \(C_1, C_2, ..., C_{k-1}\) such that \(C_i \vdash C_{i+1}\) for all \(0 \leq i < k\). 
\end{defn}

\par It is defined as a recursive function in Agda as follow: 
\begin{lstlisting}[mathescape=true,xleftmargin=.3\textwidth]
  _$\vdash^k$_-_ : (Q $\times$ $\Sigma^{e*}$) $\to$ $\mathbb{N}$ $\to$ (Q $\times$ $\Sigma^{e*}$) $\to$ Set
  (q , w$^e$) $\vdash^k$ zero  - (q' , w$^e$')
    = q $\equiv$ q' $\times$ w$^e$ $\equiv$ w$^e$'
  (q , w$^e$) $\vdash^k$ suc n - (q' , w$^e$') 
    = $\exists$[ p $\in$ Q ] $\exists$[ a$^e$ $\in$ $\Sigma^e$ ] $\exists$[ u$^e$ $\in$ $\Sigma^{e*}$ ]
      (w$^e$ $\equiv$ a$^e$ :: u$^e$ $\times$ (q , a$^e$ , u$^e$) $\vdash$ (p , u$^e$) $\times$ (p , u$^e$) $\vdash^k$ n - (q' , w$^e$'))
\end{lstlisting}

\begin{defn}
\noindent We say that \(C \vdash^* C'\) if and only
if there exists a number of chains \(n\) such that \(C \vdash^n C'\). 
\end{defn}

\par Its corresponding type is defined as follow: 
\begin{lstlisting}[mathescape=true,xleftmargin=.3\textwidth]
  _$\vdash^*$_ : (Q $\times$ $\Sigma^{e*}$) $\to$ (Q $\times$ $\Sigma^{e*}$) $\to$ Set
  (q , w$^e$) $\vdash^*$ (q' , w$^e$') = $\exists$[ n $\in$ $\mathbb{N}$ ] (q , w$^e$) $\vdash^k$ n - (q' , w$^e$')
\end{lstlisting}

\begin{defn}
\label{defn:enfa}
\noindent For any string \(w\), it is accepted by an \(\epsilon\)-NFA
if and only if there exists an \(\epsilon\)-string of \(w\)
that can take \(q_0\) to an accepting state \(q\). Therefore, the
accepting language of an \(\epsilon\)-NFA is given by the set \(\{w\ |\ \exists w^e\in
\Sigma^{e*}.\ w = to\Sigma^*(w^e) \wedge \exists q\in F.\ (q_0,w^e) \vdash^* (q,\epsilon)\}\). 
\end{defn}

\par The corresponding formalisation in Agda is as follow: 
\begin{lstlisting}[mathescape=true,xleftmargin=.3\textwidth]
  L$^{eN}$ : $\epsilon \hyphen$NFA $\to$ Language
  L$^{eN}$ nfa = $\lambda$ w $\to$ 
            $\exists$[ w$^e$ $\in$ $\Sigma^{e*}$ ] (w $\equiv$ $to\Sigma^*$ w$^e$ $\times$ ($\exists$[ q $\in$ Q ] (q $\in^d$ F $\times$ (q$_0$ , w$^e$) $\vdash^*$ (q , []))))
\end{lstlisting}

\section{Thompson's Construction}
\par Now, let us look at the translation of regular expressions to
\(\epsilon\)-NFA. The translation is defined as the function
\textbf{regexToeNFA} in \textbf{Translation/RegExp-eNFA.agda} while
the constructions of states are defined in \textbf{State.agda}. 

\begin{defn}
\label{defn:thompson}
\noindent The translation of regular expressions
to \(\epsilon\)-NFA is defined inductively as follow:
\begin{enumerate}[nolistsep]
  \item for \(\O\), we have \(M = (\{init\},\ \Sigma^e,\ \delta,\
    init,\ \O)\) and graphically, \begin{center}\includegraphics{null}\end{center}
  \item for \(\epsilon\), we have \(M = (\{init\},\ \Sigma^e,\
    \delta,\ init,\ \{init\})\) and graphically, \begin{center}\includegraphics{epsilon}\end{center}
  \item for \(a\), we have \(M = (\{init, accept\},\ \Sigma^e,\
    \delta,\ init,\ \{accept\})\) and graphically, \begin{center}\includegraphics{singleton}\end{center}
  \item suppose \(N_1 = (Q_1,\ \delta_1,\ q_{01},\ F_1)\) and \(N_2 =
    (Q_2,\ \delta_2,\ q_{02},\ F_2)\) are \(\epsilon\)-NFAs translated from the
    regular expressions \(e_1\) and \(e_2\) respectively, then
    \begin{enumerate}[nolistsep]
      \item for \((e_1\ |\ e_2)\), we have \(M = (\{init\} \cup Q_1
        \cup Q_2,\ \Sigma^e,\ \delta,\ init,\ F_1 \cup F_2)\) and
        graphically, \begin{center}\includegraphics{union}\end{center}
      \item for \(e_1\cdot e_2\), we have \(M = (Q_1 \cup \{mid\}
        \cup Q_2,\ \Sigma^e,\ \delta,\ init,\ F_2)\) and graphically, \begin{center}\includegraphics{concat}\end{center}
      \item for \(e_1^{\ *}\), we have \(M = (\{init\} \cup Q_1,\
        \Sigma^e,\ \delta,\ init,\ \{init\} \cup F_1)\) and
        graphically, \begin{center}\includegraphics{star}\end{center}
     \end{enumerate}
\end{enumerate}
\end{defn}

\par Apart from the five fields, the other fields in the record
\textit{\(\epsilon\)-NFA} are also constructed by the function. Now, let us prove the correctness of the above translation by
proving that their accepting languages are equal. The correctness
proof is defined as the function \mmb{L^R \!\approx\! L^{eN}} in
\textbf{Correctness.agda} while the detail proofs are
defined in \textbf{Correctness/RegExp-eNFA.agda}. 

\begin{thm} 
\noindent For any given regular expression, \(e\), its accepting
language is equal to the language accepted by the \(\epsilon\)-NFA
translated from \(e\) using Thompson's Construction, i.e. \(L(e) =
L(\)translated \(\epsilon\)-NFA\()\). 
\end{thm} 

\begin{proof}
\noindent We can prove the theorem by induction on regular
expressions. 

\par \noindent \textbf{Base cases.}\quad By \autoref{defn:thompson}, it is
obvious that the statement holds for \(\O\), \(\epsilon\) and
\(a\). 

\par \noindent \textbf{Induction hypothesis 1.}\quad For any two regular expressions
\(e_1\) and \(e_2\), let \(N_1 =
(Q_1,\ \delta_1,\ q_{01},\ F_1)\) and \(N_2 = (Q_2,\ \delta_2,\
q_{02},\ F_2)\) be their translated \(\epsilon\)-NFA 
respectively, we assume that \(L(e_1) = L(N_1)\) and \(L(e_2) =
L(N_2)\). 

\par \noindent \textbf{Inductive steps.}\quad There are three cases: 1)
\(e_1\ |\ e_2\), 2) \(e_1 \cdot e_2\) and 3) \(e_1^{\ *}\). 

\par \noindent 1) \textit{Case \((e_1\ |\ e_2)\)}:\quad Let \(M = (Q,\ \delta,\ q_0,\ F) = (\{init\} \cup Q_1 \cup Q_2,\
\delta,\ init,\ F_1 \cup F_2)\) be its translated \(\epsilon\)-NFA. Then for any string \(w\), 

\par 1.1) if \((e_1\ |\ e_2)\) accepts \(w\), then by
\autoref{defn:regex} and \autoref{defn:lang_union},
either i) \(e_1\) accepts \(w\) or ii) \(e_2\) accepts \(w\). Assuming case i), then by
induction hypothesis, \(N_1\) also accepts \(w\). Therefore, there
must exist an \(\epsilon\)-string of \(w\), \(w^e\), that can take \(q_{01}\)
to an accepting state \(q\) in \(N_1\). Now, consider another
\(\epsilon\)-string of \(w\), \(\epsilon w^e\), it can
take \(init\) to \(q\) in \(M\) because \(\epsilon\) can take \(init\)
to \(q_{01}\). Recall that \(q\) is an accepting
state in \(N_1\); therefore, \(q\) is also an accepting state in
\(M\) and thus, by \autoref{defn:enfa}, \(M\) accepts \(w\). The same argument also applies
to the case when \(e_2\) accepts \(w\); therefore, \(L(e_1\ |\ e_2) \subseteq L(M)\) is true; 

\par 1.2) if \(M\) accepts \(w\), then by \autoref{defn:enfa}, there must exist an
\(\epsilon\)-string of \(w\), \(w^e\), that can take \(init\) to an
accepting state \(q\) in \(M\). \(q\) must be different from \(init\) because
\(q\) is an accepting state but \(init\) is not. Now, by
\autoref{defn:thompson}, there are only two possible ways for \(init\) to reach \(q\) in \(M\), i)
via \(q_{01}\) or ii) via \(q_{02}\). Assuming case i), then \(w^e\)
must be in the form of \(aw'^e\) because \(\epsilon\) is the only alphabet that can take
\(init\) to \(q_{01}\) and thus \(w'^e\) can take \(q_{01}\) to \(q\). Furthermore, \(q\) is an
accepting state in \(M\); therefore, \(q\) is also an accepting
state in \(N_1\) and thus \(N_1\) accepts \(w\). By induction hypothesis, \(e_1\) also accepts \(w\);
therefore, we have \(w \in L(e_1\ |\ e_2)\). The same argument also
applies to case ii) and thus \(L(e_1\ |\ e_2) \supseteq L(M)\) is true; 

\par 1.3) combining 1.1 and 1.2, we have \(L(e_1\ |\ e_2) = L(M)\). 

\par \noindent 2) \textit{Case \((e_1 \cdot e_2)\)}:\quad Let \(M = (Q,\
\delta,\ q_0,\ F) = (Q_1 \cup \{mid\} \cup Q_2,\ \delta,\ q_{01},\
F_2)\) be its translated \(\epsilon\)-NFA. Then for any string
\(w\), 

\par 2.1) if \((e_1 \cdot e_2)\) accepts \(w\), then by
\autoref{defn:regex} and \autoref{defn:lang_con}, there must exist a string \(u \in L(e_1)\) and a string \(v \in L(e_2)\) such that \(w
= uv\). By induction hypothesis, \(N_1\) accepts \(u\) and \(N_2\)
accepts \(v\). Therefore, there must exist i) an \(\epsilon\)-string
of \(u\), \(u^e\), that can take \(q_{01}\) to an accepting state \(q_1\) in
\(N_1\) and ii) an \(\epsilon\)-string of \(v\), \(v^e\), that can take
\(q_{02}\) to an accepting state \(q_2\) in \(N_2\). Now, let us
consider another \(\epsilon\)-string of \(w\), \(u^e\epsilon \epsilon
v^e\), it can take \(q_{01}\) to \(q_2\) in \(M\) because \(u^e\) takes
\(q_{01}\) to \(q_1\), \(\epsilon\) takes \(q_1\) to \(mid\), another
\(\epsilon\) takes \(mid\) to \(q_{02}\) and \(v^e\) takes \(q_{02}\)
to \(q_2\). Furthermore, \(q_2\)
is also an accepting state in \(M\) because \(q_2\) is an accepting
state in \(N_2\). Therefore, \(M\) accepts \(w\)
and thus \(L(e_1 \cdot e_2) \subseteq L(M)\) is true; 

\par 2.2) if \(M\) accepts \(w\), then by \autoref{defn:enfa}, there must
exist an \(\epsilon\)-string of \(w\), \(w^e\), which can take
\(q_{01}\) to an accepting state \(q\) in \(M\). Since \(q\) is an
accepting state in \(M\); therefore, \(q\) must be in \(Q_2\). The only possible way for
\(q_{01}\) to reach \(q\) is by going through the state \(mid\). Therefore, there must exist i) an \(\epsilon\)-string, \(u^e\), that can take
\(q_{01}\) to an accepting state \(q_1\) in \(N_1\) and ii) an
\(\epsilon\)-string \(v^e\) that can take \(q_{02}\) to \(q_2\) in
\(N_1\) and \(w^e = u^e\epsilon\epsilon v^e\). Let \(u\) and \(v\) be the normal strings of \(u^e\) and
\(v^e\) respectively, then we have \(u \in L(N_1)\), \(v \in L(N_2)\) and \(w = uv\). Now, by induction
hypothesis, \(e_1\) accepts \(u\) and \(e_2\) accepts \(v\); and thus
\(e_1 \cdot e_2\) accepts \(w\). Therefore \(L(e_1 \cdot e_2) \supseteq L(M)\) is true; 

\par 2.3) combining 2.1 and 2.2, we have \(L(e_1 \cdot e_2) = L(M)\). 

\par \noindent 3) \textit{Case \(e_1^{\ *}\)}:\quad Let \(M = (Q,\ \delta,\ q_0,\
F) = (Q_1 \cup \{mid\} \cup Q_2,\ \delta,\ q_{01},\ F_2)\) be its
translated \(\epsilon\)-NFA. Then for any string \(w\), 

\par 3.1) if \((e_1^{\ *})\) accepts \(w\), then by \autoref{defn:regex}
and \autoref{defn:lang_star}, there must exist a number \(n\) such
that \(w \in L(e_1)^n\). Now, let us do induction on \(n\). 

\par \quad \textbf{Base case.} \quad When \(n = 0\), then the language
\(L^0\) can only accept the empty string \(\epsilon\); and thus \(w =
\epsilon\). From \autoref{defn:thompson}, it is obvious that \(M\)
accepts \(\epsilon\). 

\par \quad \textbf{Induction hypothesis 2.} \quad Suppose there exists a number \(k\) such that \(w
\in L(e_1)^k\), then \(w\) is also accepted by \(M\). 

\par \quad \textbf{Induction steps.} \quad When \(n = k + 1\), by
\autoref{defn:lang_con} and \autoref{defn:lang_power}, there must
exist a string \(u \in L(e_1)\) and a string \(v \in L(e_1)^k\) such
that \(w=uv\). By induction hypothesis (1), we have \(N_1\) accepts \(u\). Therefore there must exist an \(\epsilon\)-string
\(u\), \(u^e\), that can take \(q_{01}\) to an accepting state \(q\)
in \(N_1\). Since \(q\) is an accepting state; an \(\epsilon\)
alphabet can take \(q\) back to \(q_{01}\). Furthermore, by induction
hypothesis (2), \(M\) also accepts \(v\) which implies that there
exists an \(\epsilon\)-string of \(v\), \(v^e\), that can take
\(init\) to an accepting state \(p\). Since the only
alphabet that can take \(init\) to \(q_{01}\) is \(\epsilon\); therefore,
\(v^e\) must be in the form of \(\epsilon v'^e\). Now, we have proved
that there exists an \(\epsilon\) string of \(w\), \(\epsilon u^e\epsilon v'^e\), that can
take \(init\) to an accepting state \(p\) in \(M\); and thus \(M\)
accepts \(w\). Therefore \(L(e_1^{\ *}) \subseteq L(M)\) is true; 

\par 3.2) if \(M\) accepts \(w\), then by \autoref{defn:enfa}, there must exist an
\(\epsilon\)-string \(w\), \(w^e\), that can take \(init\) to an accepting
state \(q\) with \(n\) transitions in \(M\).  If \(init = q\), then \(w\) must be an empty
string. By \autoref{defn:thompson}, it is obvious that
the empty string is accepted by \(e_1^{\ *}\). If \(init \neq q\),
then there are only two possible ways for \(init\)
to reach \(q\): 1) from \(init\) to \(q\) without going back to
\(q_{01}\) from an accepting state with an \(\epsilon\), we say that this path has no loops and 2) from \(init\)
to \(q\) with at least one loop. 
\par \quad \textit{Case 1}:\quad Since \(q \neq init\), then \(w^e\) must
be in the form of \(\epsilon w'^e\). Recall that the path has no loops, it is
obvious that \(N_1\) accepts \(w\). Therefore by
induction hypothesis (1), \(e_1\) accepts \(w\) and thus \(e_1^{\ *}\)
also accepts \(w\). 
\par \quad \textit{Case 2}:\quad Since \(q \neq init\), then \(w^e\) must
be in the form of \(\epsilon w'^e\). Recall that the path has loops, we
can separate \(w'^e\) into two parts: 1) an \(\epsilon\)-string
\(u^e\) that takes \(init\) to an accepting state \(p\) without loops
and 2) an \(\epsilon\)-string \(\epsilon v^e\) that takes \(p\) to
\(q_{01}\) to \(q\) with loops. Let \(u\) and \(v\) be the normal
string of \(u^e\) and \(\epsilon v^e\) respectively, then it is obvious that
\(w = uv\). By \textit{case 1}, \(e_1\) accepts \(u\). Now, consider
the path from \(q_{01}\) to \(q\). The path must have less than \(n\)
transitions; therefore, we can prove by induction the there must exist a number \(k\) such that
\(L(e_1)^k\) accepts \(v\). Combining the above proofs, we have \(w \in
L(e_1^{\ *})\). 
\par \quad Combining \textit{Case 1} and \textit{Case 2}, we have \(w \in L(e_1^{\ *}) \supseteq L(M)\); 

\par 3.3) combining 3.1 and 3.2, we have \(L(e_1^{\ *}) = L(M)\). 

\par \noindent Therefore, by induction, \(L(e) = L(\)translated
\(\epsilon\)-NFA\()\) is true for all any regular expression \(e\). 
\end{proof}

\newpage 
\par Below is a code snippet of our formalisation of \(L^R \subseteq
L^{eN}\). 

\begin{center} \includegraphics[width=\textwidth]{thm11} \end{center}

\par Pattern matching the regular expression corresponds to the case
analysis in the proof. As shown above, there are six cases:
\(\emptyset\), \(epsilon\), \(\sigma\ a\), union, concatenation and
kleen star. The first three correpsonds to the three base
cases. Notice that the recursive calls in last three cases is equivalent to the induction hypothesis of the
proof. These cases are equivalent to the induction steps. 

\subsection{Non-deterministic Finite Automata}
\par Although the definition of NFA is very similar to that of
\(\epsilon\)-NFA, we will still define NFA
separately. The type of NFA is defined in \textbf{NFA.agda} along with
its operations and properties. 

\begin{defn}
\noindent A NFA is a 5-tuple \(M = (Q
,\ \Sigma,\ \delta,\ q_0,\ F)\), where
\begin{enumerate}[nolistsep]
  \item \(Q\) is a finite set of states;
  \item \(\Sigma\) is the set of alphabets;
  \item \(\delta\) is a mapping from \(Q \times \Sigma\) to
    \(\mathcal P \left({Q}\right)\) that defines the behaviour of the automata;
  \item \(q_0\) in \(Q\) is the initial state;
  \item \(F \subseteq Q\) is the set of accepting states. 
\end{enumerate}
\end{defn}

\par It is formalised as a record in Agda as shown below: 

\begin{lstlisting}[mathescape=true,xleftmargin=.3\textwidth]
record NFA : Set$_1$ where
  field
    Q      : Set
    $\delta$       : Q $\to$ $\Sigma$ $\to$ DecSubset Q
    q$_0$      : Q
    F      : DecSubset Q
    Q?     : DecEq Q
    |Q|-1  : $\mathbb{N}$
    It     : Vec Q (suc |Q|-1)
    $\forall$q$\in$It    : (q : Q) $\to$ (q $\in^V$ It)
    unique : Unique It
\end{lstlisting}

\par The set of alphabets \(\Sigma\) is passed into the module as a
parameter. Together with \(Q\), \(\delta\),
\(q_0\) and \(F\), these five fields correspond to the 5-tuple
NFA. The other extra fields are used in powerset construction. They
are \(Q?\) -- the decidable equality of \(Q\);
\(|Q|-1\) -- the number of states minus 1; \(It\) -- a vector containing every state in \(Q\); \(\forall q\in It\)
-- a proof that every state in \(Q\) is also in the vector
\(It\); and \(unique\) -- a proof that there is no repeating elements in
\(It\). 

\par Now, before we can define the accepting language of a given
NFA, we need to define several operations of NFA. 

\begin{defn}
\noindent A configuration is composed of a state and an alphabet from
\(\Sigma\), i.e. \(C = Q \times \Sigma\). 
\end{defn}

\begin{defn}
\noindent A move in an NFA is
represented by a binary function (\(\vdash\)) on two configurations. We say
that for all \(w \in \Sigma^*\) and \(a \in \Sigma\), \((q, aw)
\vdash (q' , w)\) if and only if \(q' \in \delta (q , a)\). 
\end{defn}

\par The binary function is defined in Agda as follow: 
\begin{lstlisting}[mathescape=true,xleftmargin=.3\textwidth]
  _$\vdash$_ : (Q $\times$ $\Sigma$ $\times$ $\Sigma^*$) $\to$ (Q $\times$ $\Sigma^*$) $\to$ Set
  (q , a , w) $\vdash$ (q' , w') = w $\equiv$ w' $\times$ q' $\in^d$ $\delta$ q a
\end{lstlisting}

\begin{defn}
\noindent Suppose \(C\) and \(C'\) are configurations. We say that \(C \vdash^0 C'\) if and only
if \(C = C'\); and \(C_0 \vdash^k C_k\) for any \(k \geq 1\) if and only if there exists a chain of
configurations \(C_1, C_2, ..., C_{k-1}\) such that \(C_i \vdash C_{i+1}\) for all \(0 \leq i < k\). 
\end{defn}

\par It is defined as a recursive function in Agda as follow: 
\begin{lstlisting}[mathescape=true,xleftmargin=.3\textwidth]
  _$\vdash^k$_-_ : (Q $\times$ $\Sigma^*$) $\to$ $\mathbb{N}$ $\to$ (Q $\times$ $\Sigma^*$) $\to$ Set
  (q , w) $\vdash^k$ zero  - (q' , w')
    = q $\equiv$ q' $\times$ w $\equiv$ w'
  (q , w) $\vdash^k$ suc n - (q' , w') 
    = $\exists$[ p $\in$ Q ] $\exists$[ a $\in$ $\Sigma$ ] $\exists$[ u $\in$ $\Sigma^*$ ]
      (w $\equiv$ a :: u $\times$ (q , a , u) $\vdash$ (p , u) $\times$ (p , u) $\vdash^k$ n - (q' , w'))
\end{lstlisting}

\begin{defn}
\noindent We say that \(C \vdash^* C'\) if and only
if there exists a number of chains \(n\) such that \(C \vdash^n C'\). 
\end{defn}

\par Its corresponding type is defined as follow: 
\begin{lstlisting}[mathescape=true,xleftmargin=.3\textwidth]
  _$\vdash^*$_ : (Q $\times$ $\Sigma^*$) $\to$ (Q $\times$ $\Sigma^*$) $\to$ Set
  (q , w) $\vdash^*$ (q' , w') = $\exists$[ n $\in$ $\mathbb{N}$ ] (q,w) $\vdash^k$ n - (q',w')
\end{lstlisting}

\begin{defn}
\noindent For any string \(w\), it is accepted by an NFA
if and only \(w\) can take \(q_0\) to an accepting state \(q\). Therefore, the
accepting language of an NFA is given by the set \(\{w\ |\ \exists q\in F.\ (q_0,w) \vdash^* (q,\epsilon)\}\). 
\end{defn}

\par The corresponding formalisation in Agda is as follow: 
\begin{lstlisting}[mathescape=true,xleftmargin=.3\textwidth]
  L$^N$ : NFA $\to$ Language
  L$^N$ nfa = $\lambda$ w $\to$ 
            $\exists$[ q $\in$ Q ] (q $\in^d$ F $\times$ (q$_0$ , w) $\vdash^*$ (q , [])))
\end{lstlisting} 


\subsection{Removing \(\epsilon\)-transitions}
\par The translation of \(\epsilon\)-NFA to NFA is defined in
\textbf{Translation/eNFA-NFA.agda}. In order to remove all the \(\epsilon\)-transitions, we need to
know whether a state \(q\) can reach another state \(q'\) with only
\(\epsilon\)-transitions. Let us begin by defining such a relation on states. 

\begin{defn}
\noindent We say that
\(q \to_\epsilon^0 q'\) if and only if
\(q\) is equal to \(q'\); and \(q \to_\epsilon^k q'\) for \(k \geq
1\) if and only if \(q\) can be transited to \(q'\) with \(k\)
\(\epsilon\)-transitions. We call this an \(\epsilon\)-path from \(q\) to \(q'\).
\end{defn}

\par It is defined as a recursive function in Agda as follow:
\begin{lstlisting}[mathescape=true,xleftmargin=.3\textwidth]
_$\to_\epsilon^k$_-_ : Q $\to$ $\mathbb{N}$ $\to$ Q $\to$ Set
q $\to_\epsilon^k$ zero - q' = q $\equiv$ q'
q $\to_\epsilon^k$ suc n - q' = $\exists$[ p $\in$ Q ] ( p $\in^d$ $\delta$ q E $\times$ p $\to_\epsilon^k$ n - q' )
\end{lstlisting}

\begin{defn}
\noindent We say that \(q \to_\epsilon^* q'\) if and only if there
exists an \(\epsilon\)-path from \(q\) to \(q'\) with \(n\) transitions, i.e. \(\exists n.\ q \to_\epsilon^n q'\). 
\end{defn}

\par The corresponding type is as follow: 
\begin{lstlisting}[mathescape=true,xleftmargin=.3\textwidth]
_$\to_\epsilon^*$_ : Q $\to$ Q $\to$ Set
q $\to_\epsilon^*$ q' = $\exists$[ n $\in$ $\mathbb{N}$ ] q $\to_\epsilon^k$ n - q'
\end{lstlisting}

\par Now we have to prove that for any two states \(q\) and \(q'\), 
\(q \to_\epsilon^* q'\) is decidable. However, it is not possible to
prove it directly because the set of natural numbers is
infinite. Therefore, we will introduce an algorithm that computes the
\(\epsilon\)-closure of a state. The \(\epsilon\)-closure of a state
\(q\), \(\epsilon\hyphen closure(q)\) should contain all the states
that are reachable from \(q\) with only \(\epsilon\)-transitions. We will prove that for any two states
\(q\) and \(q'\), \(q \to_\epsilon^* q'\) if and only if \(q' \in \epsilon\hyphen closure(q)\). By proving that they are equivalent, we
will have proved the decidability of \(q \to_\epsilon^* q'\). 

\begin{defn}
\noindent For any given state \(q\), \(\epsilon\hyphen closure(q)\) is obtained by: 
\begin{enumerate}[nolistsep]
  \item put \(q\) into a subset \(S\), i.e. \(S = \{q\}\)
  \item loop for \(|Q| - 1\) times: 
  \begin{enumerate}
    \item for every state \(p\) in \(S\), if \(\epsilon\) can take
      \(p\) to another state \(r\), i.e. \(r \in
        \delta (p,\epsilon)\), then put \(r\) into \(S\). 
  \end{enumerate}
  \item the result subset \(S\) is the \(\epsilon\)-closure of \(q\)
\end{enumerate}
\end{defn}

\begin{lem}
\noindent For any two states \(q\) and \(q'\), \(q \to_\epsilon^* q'\)
if and only if \(q' \in \epsilon\hyphen closure(q)\). 
\end{lem}

\begin{proof}
\noindent We have to prove for both directions. 
\par \noindent 1) If \(q \to_\epsilon^* q'\), then there must exist a
number \(n\) such that \(\epsilon^n\) can take \(q\) to \(q'\). If
\(n < |Q|\), then it is obvious that \(q' \in
\epsilon\hyphen closure(q)\) is true. If \(n \geq |Q|\), the
\(\epsilon\)-path from \(q\) to \(q'\) must have loop(s) inside. By
removing the loop(s), the equivalent \(\epsilon\)-path must have at
most \(|Q|-1\) \(\epsilon\)-transitions. Therefore, \(q' \in
\epsilon\hyphen closure(q)\) is true. 

\par \noindent 2) If \(q' \in \epsilon\hyphen closure(q)\), it is obvious
that \(q \to_\epsilon^{|Q|-1} q'\) must be true and thus \(q
\to_\epsilon^* q'\) is true. 
\end{proof}

\par Since we have proved that they are equivalent; therefore the
decidability of \(q \to_\epsilon^* q'\) follows. Now, let us define
the translation of \(\epsilon\)-NFA to NFA using \(q \to_\epsilon^* q'\). 
 
\begin{defn}
\label{defn:remove_epsilon}
\noindent For a given \(\epsilon\)-NFA, \((Q,\ \delta,\
q_0,\ F)\), its translated NFA will be \((Q,\ \delta',\ q_0,
F')\) where
\begin{itemize}[nolistsep]
  \item \(\delta'(q,a) = \delta (q,a) \cup \{q'\ |\ \exists p.\
      q \to_\epsilon^* p \wedge q' \in \delta (p,a)\}\);
  \item \(F' = F \cup \{q\ |\ \exists p\in F.\ q \to_\epsilon^* p\}
    \). 
\end{itemize}
\end{defn}

\par Now, let us prove the correctness of the above translation by proving their accepting languages
are equal. The correctness proofs can be found in \textbf{Correctness/eNFA-NFA.agda}.

\begin{thm}
\noindent For any \(\epsilon\)-NFA, its accepting language is equal to
the language accepted by its translated NFA, i.e. \(L(\epsilon\)-NFA\()
= L(\)translated NFA\()\). 
\end{thm}

\begin{proof}
\noindent Let the \(\epsilon\)-NFA be \(\epsilon N = (Q,\ \delta,\ q_0,\
F)\), and its translated NFA be \(N = (Q,\ \delta',\ q_0,\ F')\)
according to \autoref{defn:remove_epsilon}. To
prove the theorem, we have to prove that \(L(\epsilon N) \subseteq
L(N)\) and \(L(\epsilon N) \supseteq L(N)\). Then for any string \(w\), 

\par \noindent 1) if \(\epsilon N\) accepts \(w\), then there must exist an
\(\epsilon\)-string of \(w\), \(w^e\), that can take \(q_0\) to an
accepting state \(q\) with \(n\) transitions. There are three
possibilities: a) the last transition in the path is not an
\(\epsilon\)-transition, b) the path is divided into three parts, the
first part from \(q_0\) to a state \(p\) with less than \(n\)
transitions; the second part from \(p\) to a state \(p_1\) with an
alphabet \(a\) and the third part from \(p_1\) to \(q\) with only
\(\epsilon\)-transitions and c) the path from \(q_0\)
to \(q\) consists of only \(\epsilon\)-transitions.

\par \textit{Case a}:\quad There must exist a state \(p\), an
\(\epsilon\)-string \(u^e\), and an alphabet \(a\) such that \(u^e\)
can take \(q_0\) to \(p\) in less than \(n\) transitions, \(q \in
\delta(p,a)\) and \(w^e = u^ea\). Since that path from \(q_0\) to
\(p\) is less than \(n\) transitions; therefore, we can prove by
induction that the normal string of \(u^e\), \(u\), can take
\(q_0\) to \(p\) in \(N\). Furthermore, \(w = ua\); therefore, \(w\) can
take \(q_0\) to \(q\) in \(N\) and thus \(N\) accepts \(w\). 

\par \textit{Case b}:\quad There must exist two states \(p\) and \(p_1\), an
\(\epsilon\)-string of \(w\), \(u^e\), and an alphabet \(a\) such
that \(u^e\) can take \(q_0\) to \(p\) in less than \(n\) transitions,
\(p_1 \in \delta(p,a)\), \(p_1 \to_\epsilon^* q\) and \(w = ua\). Since that path from \(q_0\) to
\(p\) is less than \(n\) transitions; therefore, we can prove by
induction that the normal string of \(u^e\), \(u\), can take
\(q_0\) to \(p\) in \(N\). Furthermore, \(p_1\) must be an accepting state in \(N\)
because it can be transited to an accepting state \(q\) with only
\(\epsilon\)-transitions. Therefore, \(w\) can take \(q_0\) to an
accepting state \(p_1\) in \(N\) and thus \(N\) accepts \(w\). 

\par \textit{Case c}:\quad If the path from \(q_0\)
to \(q\) consists of only \(\epsilon\)-transitions, then \(q_0
\to_\epsilon^* q\) is true; therefore, \(q_0\) is also an accepting
state in \(N\). Furthermore, the accepted string consists of \(\epsilon\) only;
therefore, \(w = \epsilon\). It is obvious that \(N\) accepts \(\epsilon\). 

\par \noindent 2) if \(N\) accepts \(w\), then \(w\) must be able to take \(q_0\) to an
accepting state \(q\). Since \(q\) is an accepting state, then \(q\)
is also an accepting state in \(\epsilon N\) or there exists a state 
\(p\) such that \(q \to_\epsilon^* p\) and \(p \in F\). For the former
case, since \(w\) is also an \(\epsilon\)-string of itself; therefore,
it is obvious that \(\epsilon N\) accepts \(w\). For the latter
case, suppose the path from \(q\) to \(p\) has \(n\)
\(\epsilon\)-transitions, the consider another \(\epsilon\)-string of
\(w\), \(w\epsilon^n\), it can take \(q_0\) to \(p\) in
\(\epsilon N\). Therefore \(\epsilon N\) accepts \(w\). 
\end{proof}

\section{Deterministic Finite Automata}
\par The type of DFA is defined in \textbf{DFA.agda} along with its
operations and properties. 

\begin{defn}
\noindent A DFA is a 5-tuple \(M = (Q
,\ \Sigma,\ \delta,\ q_0,\ F)\), where
\begin{enumerate}[nolistsep]
  \item \(Q\) is a finite set of states;
  \item \(\Sigma\) is the set of alphabets;
  \item \(\delta\) is a mapping from \(Q \times \Sigma\) to \(Q\) that defines the behaviour of the automata;
  \item \(q_0\) in \(Q\) is the initial state;
  \item \(F \subseteq Q\) is the set of accepting states. 
\end{enumerate}
\end{defn}

\par It is formalised as a record in Agda as shown below: 

\begin{lstlisting}[mathescape=true,xleftmargin=.15\textwidth]
record DFA : Set$_1$ where
  field
    Q      : Set
    $\delta$       : Q $\to$ $\Sigma$ $\to$ DecSubset Q
    q$_0$      : Q
    F      : DecSubset Q
    _$\approx$_     : Q $\to$ Q $\to$ Set
    $\approx\hyphen$isEquiv : IsEquivalence _$\approx$_
    $\delta\hyphen$lem    : $\forall$ {q} {p} a $\to$ q $\approx$ p $\to$ $\delta$ q a $\approx$ $\delta$ p a
    F$\hyphen$lem   : $\forall$ {q} {p}   $\to$ q $\approx$ p $\to$ q $\in^d$ F $\to$ p $\in^d$ F
\end{lstlisting}

\par The set of alphabets \(\Sigma\) is passed into the module as a
parameter. Together with \(Q\), \(\delta\),
\(q_0\) and \(F\), these five fields correspond to the 5-tuple
DFA. The other extra fields are used in proving its decidability. They
are \(\_\approx\_\) -- an equivalence relation on states;
\(\approx\hyphen isEquiv\) -- a proof that the relation \(\approx\) is
an equivalence relation; \(\delta\hyphen lem\) -- a proof that for any
alphabet \(a\) and any two states \(q\) and \(p\), if \(q \approx p\)
then \(\delta (q,a) \approx \delta (p,a)\); and \(F\hyphen lem\) -- a
proof that for any two states \(q\) and \(p\), if \(q \approx p\) and
\(q\) is an accepting state, then \(p\) is also an accepting state. 

\par Now, before we can define the accepting language of a given
DFA, we need to define several operations of DFA. 

\begin{defn}
\noindent A configuration is composed of a state and an alphabet from
\(\Sigma\), i.e. \(C = Q \times \Sigma\). 
\end{defn}

\begin{defn}
\noindent A move in an DFA is
represented by a binary function (\(\vdash\)) on two configurations. We say
that for all \(w \in \Sigma^*\) and \(a \in \Sigma\), \((q, aw)
\vdash (q' , w)\) if and only if \(q' = \delta (q , a)\). 
\end{defn}

\par The binary function is defined in Agda as follow: 
\begin{lstlisting}[mathescape=true,xleftmargin=.1\textwidth]
  _$\vdash$_ : (Q $\times$ $\Sigma$ $\times$ $\Sigma^*$) $\to$ (Q $\times$ $\Sigma^*$) $\to$ Set
  (q , a , w) $\vdash$ (q' , w') = w $\equiv$ w' $\times$ q' $\approx$ $\delta$ q a
\end{lstlisting}

\begin{defn}
\noindent Suppose \(C\) and \(C'\) are configurations. We say that \(C \vdash^0 C'\) if and only
if \(C = C'\); and \(C_0 \vdash^k C_k\) for any \(k \geq 1\) if and only if there exists a chain of
configurations \(C_1, C_2, ..., C_{k-1}\) such that \(C_i \vdash C_{i+1}\) for all \(0 \leq i < k\). 
\end{defn}

\par It is defined as a recursive function in Agda as follow: 
\begin{lstlisting}[mathescape=true,xleftmargin=.1\textwidth]
  _$\vdash^k$_-_ : (Q $\times$ $\Sigma^*$) $\to$ $\mathbb{N}$ $\to$ (Q $\times$ $\Sigma^*$) $\to$ Set
  (q , w) $\vdash^k$ zero  - (q' , w')
    = q $\equiv$ q' $\times$ w $\equiv$ w'
  (q , w) $\vdash^k$ suc n - (q' , w') 
    = $\exists$[ p $\in$ Q ] $\exists$[ a $\in$ $\Sigma$ ] $\exists$[ u $\in$ $\Sigma^*$ ]
      (w $\equiv$ a :: u $\times$ (q , a , u) $\vdash$ (p , u) 
       $\times$ (p , u) $\vdash^k$ n - (q' , w'))
\end{lstlisting}

\begin{defn}
\noindent We say that \(C \vdash^* C'\) if and only
if there exists a number of chains \(n\) such that \(C \vdash^n C'\). 
\end{defn}

\par Its corresponding type is defined as follow: 
\begin{lstlisting}[mathescape=true,xleftmargin=.1\textwidth]
  _$\vdash^*$_ : (Q $\times$ $\Sigma^*$) $\to$ (Q $\times$ $\Sigma^*$) $\to$ Set
  (q , w) $\vdash^*$ (q' , w') = $\exists$[ n $\in$ $\mathbb{N}$ ] (q,w) $\vdash^k$ n - (q',w')
\end{lstlisting}

\begin{defn}
\noindent For any string \(w\), it is accepted by an DFA
if and only \(w\) can take \(q_0\) to an accepting state \(q\). Therefore, the
accepting language of an DFA is given by the set \(\{w\ |\ \exists q\in F.\ (q_0,w) \vdash^* (q,\epsilon)\}\). 
\end{defn}

\par The corresponding formalisation in Agda is as follow: 
\begin{lstlisting}[mathescape=true,xleftmargin=.1\textwidth]
  L$^D$ : DFA $\to$ Language
  L$^D$ dfa = $\lambda$ w $\to$ 
            $\exists$[ q $\in$ Q ] (q $\in^d$ F $\times$ (q$_0$ , w) $\vdash^*$ (q , [])))
\end{lstlisting} 


\section{Powerset Construction}
\par The translation of NFA to DFA is defined as the function
\textbf{powerset-construction} in the file \textbf{Translation/NFA-DFA.agda}. 

\begin{defn}
\label{defn:powerset}
\noindent For any given NFA, \((Q,\ \delta,\ q_0,\ F)\), its
translated DFA will be \((\mathcal P \left({Q}\right),\ \delta',\ \{q_0\},\ F')\) where
\begin{itemize}[nolistsep]
  \item \(\delta'(qs,a) = \{q'\ |\ \exists q\in qs.\ q' \in \delta (q,a)\}\);
  \item \(F' = \{qs\ |\ \exists q\in F.\ q \in qs\}\). 
\end{itemize}
\end{defn}

\par In Agda, the set \(\mathcal P \left({Q}\right)\) is defined as
the set of decidable subsets, i.e. \(Q' = DecSubset Q\). Therefore,
the decidability of the set \(\delta'(qs,a)\) and \(F'\) must be
proved using the vector representation of \(Q\). The corresponding
proofs are defined in the module \textbf{Powerset-Construction} in the
same file. 

\par Now, before proving that their accepting languages are equal, we 
need to prove the following lemmas which can be found in
\textbf{Correctness/NFA-DFA.agda}. 

\begin{lem}
\label{lem:nfa<dfa}
\noindent Let a NFA be \(N = (Q,\ \delta,\ q_0,\ F)\) and its
translated DFA be \(D = (\mathcal P \left({Q}\right),\ \delta',\ {q_0},
F')\) according to \autoref{defn:powerset}. For any string \(w\), if \(w\) can take \(q_0\) to a state
\(q\) with \(n\) transitions in \(N\), then there must exist a subset \(qs\) such that \(q
\in qs\) and \(w\) can take \(\{q_0\}\) to \(qs\) in \(D\),
i.e. \(\forall w.\exists q.\exists n.\ (q_0,w) \vdash^n (q,w) \Rightarrow
\exists qs.\ q \in qs \wedge ({q_0},w) \vdash^* (qs,\epsilon)\). 
\end{lem}

\par The proof is defined as the function \mmb{lem_1} under the module
\mmb{L^N\!\subseteq L^D}. 

\begin{proof}
\noindent We can prove the lemma by induction on \(n\).
\par \noindent \textbf{Base case.}\quad If \(n = 0\), then \(q_0 = q\)
and \(w = \epsilon\). It is obvious that the statement holds.

\par \noindent \textbf{Induction hypothesis.}\quad For any string \(w\), if \(w\) can take \(q_0\) to a state
\(q\) with \(k\) transitions in \(N\), then there exists a subset \(qs\) such that \(q
\in qs\) and \(w\) can take \(\{q_0\}\) to \(qs\) in \(D\). 

\par \noindent \textbf{Induction step.}\quad If \(n = k + 1\), then
\(w\) can take \(q_0\) to a state \(q\) by \(k + 1\) transitions. Let
\(w = w'a\) where \(a\) is an alphabet, then \(w'\) can take \(q_0\)
to a state \(p\) by \(k\) transitions and \(q \in \delta(p,a)\). By induction hypothesis, there must exist a subset \(ps\) such that \(p
\in ps\) and \(w'\) can take \(\{q_0\}\) to \(ps\) in \(D\). Furthermore, since \(q \in \delta(p,a)\), then \(a\) must be
able to take \(ps\) to a subset \(qs\) where \(q \in qs\). Therefore,
there exists a subset \(qs\) such that \(q \in qs\) and \(w\) can take
\(\{q_0\}\) to \(qs\); and thus the statement is true. 
\end{proof}

\begin{lem}
\label{lem:nfa>dfa}
\noindent Let a NFA be \(N = (Q,\ \delta,\ q_0,\ F)\) and its
translated DFA be \(D = (\mathcal P \left({Q}\right),\ \delta',\ {q_0},
F')\) according to \autoref{defn:powerset}. For any string \(w\), any
number \(n\) and any two states \(qs\) and \(ps\) in \(\mathcal P \left({Q}\right)\), if
\((qs,w) \vdash^n (ps,\epsilon)\) then \(ps =
\{p\ |\ \exists q\in qs.\ (q,w) \vdash^n (p,\epsilon)\}\). 
\end{lem}

\par The proof is defined as the function \mmb{lem_2} under the module
\mmb{L^N\!\supseteq L^D}. 

\begin{proof}
\noindent We can prove the lemma by induction on \(n\). 

\par \noindent \textbf{Base case.}\quad If \(n = 0\), then \(qs = ps\)
and \(w = \epsilon\). Then for any state \(p\) in \(Q\), 
\par \noindent  1) if \(p \in ps\), then \(p\) is also in \(qs\). It is obvious that
\((p,\epsilon) \vdash^0 (p, \epsilon)\) is true; therefore, \(p \in
\{p\ |\ \exists q\in qs.\ (q,w) \vdash^0 (p,\epsilon)\}\); 
\par \noindent  2) if \(p \in \{p\ |\ \exists q\in qs.\ (q,w) \vdash^0
(p,\epsilon)\}\), then \(p\) must be in \(qs\) and thus \(p \in ps\). 

\par \noindent \textbf{Induction hypothesis.}\quad For any string \(w\) and any
two states \(qs\) and \(ps\) in \(\mathcal P \left({Q}\right)\), if
\((qs,w) \vdash^k (ps,\epsilon)\) then \(ps =
\{p\ |\ \exists q\in qs.\ (q,w) \vdash^k (p,\epsilon)\}\). 

\par \noindent \textbf{Induction step.}\quad If \(n = k + 1\), then
\(w\) must be able to take \(qs\) to \(ps\) with \(k + 1\)
transitions in \(D\). Therefore there must exist an alphabet \(a\)
that can take \(qs\) to a subset \(rs\),
i.e. \(rs = \delta'(qs,a)\); and a string \(u\) that can take \(rs\) to
\(ps\) with \(k\) transitions. By induction hypothesis, we have \(ps =
\{p\ |\ \exists r\in rs.\ (r,u) \vdash^k (p,\epsilon)\}\). Then for any state \(p\) in \(Q\), 

\par \noindent  1) if \(p \in ps\), then there must exist a state \(r \in rs\)
such that \((r,u) \vdash^k (p,\epsilon)\). Since \(rs =
\delta'(qs,a)\); therefore, \(r \in \delta'(qs,a)\) and thus there
exists a state \(q \in qs\) such that \(r \in \delta (q,a)\). Therefore,
\((q,w) \vdash^{k+1} (p,\epsilon)\) is true and thus \(p \in \{p\ |\ \exists q\in qs.\ (q,w) \vdash^{k+1}
(p,\epsilon)\}\). 

\par \noindent  2) if \(p \in \{p\ |\ \exists q\in qs.\ (q,w) \vdash^{k+1}
(p,\epsilon)\}\), then there exists a state \(q \in qs\) such that \((q,w) \vdash^{k+1}
(p,\epsilon)\). Also, there must exist a state \(r \in Q\) such that
\(r \in \delta (q,a)\) and the string \(u\) can take \(r\) to \(p\) in
\(N\). Since, \(q
\in qs\) and \(r \in \delta (q,a)\); therefore, \(r \in \delta'(qs,a)
= rs\) and thus \(p \in \{p\ |\ \exists r\in rs.\ (r,u) \vdash^k
(p,\epsilon)\} = ps\). 
\end{proof}

\par Now, by using the above lemmas, we can prove the correctness of
the translation by proving that their accepting languages are
equal. The correctness proof is defined as the function \mmb{L^N
  \!\approx\! L^D} in \textbf{Correctness.agda} while the detail
proofs are defined in \textbf{Correctness/NFA-DFA.agda}. 

\begin{thm}
\noindent For any NFA, its accepting language is equal to
the language accepted by its translated DFA, i.e. \(L(\)NFA\()
= L(\)translated DFA\()\). 
\end{thm}

\begin{proof}
\noindent For a given NFA, \(N = (Q,\ \delta,\ q_0,\ F)\), let its
translated DFA be \(D = (\mathcal P \left({Q}\right),\ \delta',\
{q_0},\ F')\). To
prove the theorem, we have to prove that \(L(N) \subseteq L(D)\) and
\(L(N) \supseteq L(D)\). For any string \(w\), 

\par \noindent 1) if \(N\) accepts \(w\), then \(w\) can take \(q_0\) to an
accepting state \(q\) with \(n\) transitions in \(N\). By
\autoref{lem:nfa<dfa}, there must exist a subset \(qs\) such that
\(q \in qs\) and \(w\)
can take \(\{q_0\}\) to \(qs\) in \(D\). Since \(q\) is
an accepting state; therefore, \(qs\) is also an accepting state in
\(D\) and thus \(D\) accepts \(w\). 

\par \noindent 2) if \(D\) accepts \(w\), then \(w\) can take
\(\{q_0\}\) to an accepting state \(qs\) in \(D\) with \(n\)
transitions. Since \(qs\) is an accepting state, therefore, there must
exist a state \(q\) in \(Q\) such that \(q \in qs\) and \(q\) is also
an accepting state in \(N\). Assuming \(w\) cannot take \(q_0\) to
\(q\) in \(N\) for any number of transitions, then by
\autoref{lem:nfa>dfa}, \(qs = \emptyset\) and thus \(q \notin qs\) which contradicts the assumption that \(q \in
qs\). Therefore, \(w\) must be able to take \(q_0\) to \(q\) in \(N\) and thus \(N\) accepts \(w\). 
\end{proof}


\section{Decidability of DFA and Regular Expressions}
\par Recall that the accepting language of a DFA is given by the set
\(\{w\ |\ \exists q\in F.\ (q_0,w) \vdash^* (q,\epsilon)\}\), which is
equivalent to the set \(\{w\ |\ \exists q\in F.\ \exists n.\ (q_0,w) \vdash^n (q,\epsilon)\}\). 
Its decidability cannot be proved directly because the set of natural
numbers is infinite. Therefore, we have to
introduce another representation and prove
that they are equivalent. The representation and the related lemmas are
defined under the module \textbf{DFA-Operations} in \textbf{DFA.agda}. 

\begin{defn}
\noindent We define a function \(\delta^*(q,w)\) that takes a state \(q\) and
a string \(w\) as the arguments and returns a state \(p\) after
running the DFA. It is defined recursively as follow: 
\begin{lstlisting}[mathescape=true,xleftmargin=.3\textwidth]
$\delta^*$ : Q $\to$ $\Sigma^*$ $\to$ Q
$\delta^*$ q [] = q
$\delta^*$ (a :: w) = $\delta^*$ ($\delta$ q a) w
\end{lstlisting}
\end{defn}

\begin{defn}
\noindent We define \(\delta_0(w)\) as the function that runs the DFA
from \(q_0\) with a string \(w\). 
\begin{lstlisting}[mathescape=true,xleftmargin=.3\textwidth]
$\delta_0$ : $\Sigma^*$ $\to$ Q
$\delta_0$ w = $\delta^*$ $q_0$ w
\end{lstlisting}
\end{defn}

\par Now, before proving that the two representations are equivalent, we have to prove the following lemmas. 

\begin{lem}
\label{lem:dec_iff}
\noindent For any state \(q\) and any string \(w\), \(\delta^*(q,w) \in F\) if and only
if \(\exists q'\in F.\ \exists n.\ (q,w) \vdash^n (q',\epsilon)\).
\end{lem}

\begin{proof}
\noindent We have to prove for both directions. 

\par \noindent 1) If \(\delta^*(q,w) \in F\), we do induction on \(w\).
\par \textbf{Base case.}\quad If \(w = [\ ]\), then \(\delta^*(q,[\ ]) =
q \in F\). Therefore, the statement holds.
\par \textbf{Induction hypothesis.}\quad For any \(q\) and \(w'\), if
\(\delta^*(q,w') \in F\), then \(\exists q'\in F.\ \exists n.\ (q,w') \vdash^n (q',\epsilon)\).
\par \textbf{Induction step.}\quad If \(w = aw'\), then
\(\delta^*(q,aw') = \delta^*(\delta(q,a),w')\). By induction
hypothesis, there exists a state \(q' \in F\) and a number \(n\) such
that \((\delta(q,a),w') \vdash^n (q',\epsilon)\). It is equivalent to
\((q,aw') \vdash^{n+1} (q',\epsilon)\) and thus, the statement holds.

\par \noindent 2) If \(\exists q'\in F.\ \exists n.\ (q,w') \vdash^n
(q',\epsilon)\), then we do induction on \(n\).
\par \textbf{Base case.}\quad If \(n = 0\), then \(q' = q\) and \(w =
[\ ]\). Therefore, \(\delta^*(q,[\ ]) = q = q' \in F\). 
\par \textbf{Induction hypothesis.}\quad For any state \(q\) and
string \(w\), if
there exists another state \(q'\) and a number \(k\) such that \(q' \in F \wedge (q,w) \vdash^k
(q',\epsilon)\), then \(\delta^*(q,w) \in F\). 
\par \textbf{Induction step.}\quad If \(n = k + 1\), then there
exists a state \(q' \in F\) and a number \(k\) such that \((q,aw) \vdash^{k+1}
(q',\epsilon)\). It is equivalent to \((\delta(q,a),w) \vdash^k
(q',\epsilon)\). By induction hypothesis, we have
\(\delta^*(\delta(q,a),w) \in F\) and thus \(\delta^*(q,aw) \in F\). 
\end{proof} 

\par Now, we can prove that the two representations are equivalent. 

\begin{lem}
\label{lem:dec_iff2}
\noindent For any string
\(w\), \(\delta_0 (w) \in F\) if and only if \(\exists q\in F.\ (q_0,w) \vdash^* (q,\epsilon)\).
\end{lem}

\par The proof is defined as the function \mb{\delta_0\!-lem_1}. 

\begin{proof}
\noindent Since \(\delta_0(w) = \delta^*(q_0,w)\); therefore, by
\autoref{lem:dec_iff}, \(\delta_0(w) \in F\) if and only if \(\exists
q\in F.\ (q_0,w) \vdash^* (q,\epsilon)\). 
\end{proof}

\par Now, we can prove that the accepting language of a given DFA is
decidable by using the above theorems. The proof is
defined as the function \textbf{Dec-L\(^D\)} in \textbf{DFA.agda}. 

\begin{thm}
\noindent For any DFA, its accepting language is decidable,
i.e. \(\forall w.\ w \in L(\)DFA\()\) is decidable. 
\end{thm}

\begin{proof}
\noindent Since the language of DFA is given by the set \(\{w\ |\
\exists q\in F.\ (q_0,w) \vdash^* (q,\epsilon)\}\), which, by
\autoref{lem:dec_iff2}, is equal to the set \(\{w\ |\ \delta_0(w) \in
F\}\). Since \(F\) is a decidable subset; therefore, the set is also
decidable. 
\end{proof}

\par Since we have also proved that the accepting language of regular
expressions and DFA are equal; therefore, the accepting language of
regular expression must also be decidable. The proof is defined as the
function \textbf{Dec-L\(^R\)} in \textbf{RegExp-Decidability.agda}. 

\begin{thm}
\noindent For any given regular expression, \(e\), its accepting language is
decidable, i.e. \(\forall w.\ w \in L(\)e\()\) is decidable. 
\end{thm}

\begin{proof}
\noindent Since \(L(e) = L(translated\) DFA\()\) and the language
accepted by a DFA is decidable; therefore, \(L(e)\) is also decidable. 
\end{proof}

\subsection{Minimising DFA}
\par There are two procedures in minimising a DFA: 1) removing all the
unreachable states to construct a RDFA and 2) perform quotient
construction on the RDFA to build a MDFA. 

\subsubsection{Removing unreachable states}
\par The translation of DFA to RDFA is defined under the module
\textbf{Remove-Inaccessible-States} in
\textbf{Translation/DFA-MDFA.agda}. Let us begin by defining the
reachability of a state.

\begin{defn}
\noindent For a given DFA, \((Q,\ \delta,\ q_0,\ F)\), we say that a state \(q\) is reachable if and
only if there exists a string \(w\) that can take \(q_0\) to \(q\), 
\end{defn} 

\par It is defined in Agda as follow:
\begin{lstlisting}[mathescape=true,xleftmargin=.3\textwidth]
Reachable : Q $\to$ Set
Reachable q = $\exists$[ w $\in$ $\Sigma^*$ ] (q$_0$ , w) $\vdash^*$ (q , [])
\end{lstlisting}

\begin{defn}
\noindent For a given DFA, \((Q,\ \delta,\ q_0,\ F)\), we define \(Q^R\) as a
subset of \(Q\) such that \(Q^R\) contains all and only the reachable states in \(Q\). 
\end{defn}

\par The set \(Q^R\) is defined in Agda as follow:
\begin{lstlisting}[mathescape=true,xleftmargin=.3\textwidth]
data Q$^R$ : Set where
  reach : $\forall$ q $\to$ Reachable q $\to$ Q$^R$
\end{lstlisting}

\par There are some problems regarding this formalisation of \(Q^R\),
they will be discussed in Section 6 in details. Now, we can define the
translation of DFA to RDFA.

\begin{defn}
\label{defn:unreachable}
\noindent For any given DFA, \((Q,\ \delta,\ q_0,\ F)\), its
translated RDFA will be \((Q^R,\ \delta,\ q_0,\ Q^R \cap F)\). 
\end{defn}

\par Now, let us prove the correctness of the translation by proving
that their accepting languages are equal. The correctness proof can be
found in the module \textbf{Remove-Inaccessible-States-Proof} in \textbf{Correctness/DFA-MDFA.agda}. 

\begin{thm}
\noindent For any DFA, its accepting language is equal to
the language accepted by its translated RDFA, i.e. \(L(\)DFA\()
= L(\)translated RDFA\()\). 
\end{thm}

\begin{proof}
\noindent Let the DFA be \(D = (Q,\ \delta,\ q_0,\ F)\) and is
translated RDFA be \(R = (Q^R,\ \delta,\ q_0,\ Q^R \cap F)\) according
to \autoref{defn:unreachable}. To prove the theorem, we have to prove
that \(L(D) \subseteq L(R)\) and \(L(D) \supseteq L(R)\). For any
string \(w\), 

\par \noindent 1) if \(D\) accepts \(w\), then \(w\) must be able to
take \(q_0\) to an accepting state \(q\). This implies that all the states in
the path must be reachable; therefore, this path is also valid in
\(R\) and thus \(R\) accepts \(w\); 

\par \noindent 2) if \(R\) accepts \(w\), then the path from \(q_0\)
to the accepting state \(q\) must be also valid in \(D\); therefore,
\(D\) also accepts \(w\). 
\end{proof}


\subsubsection{Quotient construction}
\par Now, we need to perform quotient construction on the newly
constructed RDFA. The construction of quotient set can be found in
\textbf{Quotient.agda}. Now, let us being by defining a
binary relation on states that will be used to construct the quotient set. 

\begin{defn}
\noindent Suppose we have a DFA \((Q,\ \delta,\ q_0,\ F)\), then for any two states \(p\) and \(q\), we say
the \(p \sim q\) if and only if for any string \(w\),
\(w\) cannot distinguish \(p\) and \(q\), i.e. \(p \sim q = \forall w.\
\delta^*(p,w) \in F \Leftrightarrow \delta^*(q,w) \in F\)
\end{defn}

\par It is easy to show that the relation is an equivalence
relation. Now, we have to prove that \(p \sim q\) is decidable. One
possible method is to use the Table-filling algorithm. However, the
formalisation of this algorithm and its correctness proofs has not
been completed. Therefore, in the following parts, we will assume that
\(p \sim q\) is decidable. Now, let us construct the quotient set by using the above
relation of states. 

\begin{defn}
\noindent For a state \(p\) of a given DFA, its equivalence
class is a subset of all indistinguishable states of \(p\), i.e. \(\llangle p
\rrangle = \{q\ |\ p \sim q\}\). 
\end{defn}

\par From the above definition, we observe that a equivalence class
is a subset of the set of states of a given DFA. The corresponding formalisation in
Agda is as follow, note that \(Dec\hyphen\!\sim\) is the decidability of
\(\sim\):
\begin{lstlisting}[mathescape=true,xleftmargin=.3\textwidth]
$\llangle$_$\rrangle$ : Q $\to$ DecSubset Q
$\llangle$ p $\rrangle$ q with Dec$\hyphen\!\sim$
... | yes _ = true
... | no  _ = false
\end{lstlisting}

\begin{defn}
\noindent For a given DFA, \((Q,\ \delta,\ q_0,\ F)\), we define
\(Q/\!\sim\) as the set of all equivalence classes of
\(\sim\) on \(Q\). 
\end{defn}

\par It is defined in Agda as follow: 
\begin{lstlisting}[mathescape=true,xleftmargin=.3\textwidth]
data Quot-Set : Set where
  class : $\forall$ qs $\to$ $\exists$[ q $\in$ Q ] (qs $=^d$ $\llangle q \rrangle$) $\to$ Quot-Set
\end{lstlisting}

\par Now, we can define the translation of RDFA to MDFA which is
defined under the module \textbf{Quotient-Construct } in \textbf{Translation/DFA-MDFA.agda}. 
\begin{defn}
\label{defn:quotient}
\noindent For any given RDFA, \((Q,\ \delta,\ q_0,\ F)\), its
translated MDFA will be \((Q/\!\sim,\ \delta',\ \llangle q_0 \rrangle,\ F')\) where
\begin{itemize}[nolistsep]
\item \(\delta'(q,a) = \llangle \delta(q,a) \rrangle\); and
\item \(F' = \{\llangle q \rrangle\ |\ q \in F\}\).
\end{itemize}
\end{defn}

\par Now, before proving the translation is correct, we have to first prove
the following lemmas. The theorems and proofs below can be found in
the module \textbf{Quotient-Construction-Proof} in \textbf{Correctness/DFA-MDFA.agda}. 

\begin{lem}
\label{lem:rdfa<mdfa}
\noindent Let a RDFA be \(R = (Q,\ \delta,\ q_0,\ F)\) and its
translated MDFA be \(M = (Q/\!\sim,\ \delta',\ \llangle q_0 \rrangle,\
F')\) according to \autoref{defn:quotient}. For any state \(q\) in \(Q\), if a string \(w\) can take \(q\) to another state \(q'\) with \(n\) transitions
in \(R\), then \(w\) can take \(\llangle q \rrangle\) to \(\llangle q'
\rrangle\) with \(n\) transitions in \(M\). 
\end{lem}

\begin{proof}
\noindent We can prove the lemma by induction on \(n\).
\par \noindent \textbf{Base case.}\quad If \(n = 0\), then \(q = q'\)
and \(w = \epsilon\). It is obvious that the statement holds.

\par \noindent \textbf{Induction hypothesis.}\quad For any state \(q\)
in \(Q\), if a string \(w\) can take \(q\) to another state \(q'\)
with \(k\)  transitions
in \(R\), then \(w\) can take \(\llangle q \rrangle\) to \(\llangle q'
\rrangle\) with \(k\) transitions in \(M\). 

\par \noindent \textbf{Induction step.}\quad If \(n = k + 1\), then
there must exist a string in the form of \(aw\) that can take \(q\)
to \(q'\) with \(k + 1\) transitions. This implies that there exists a
state \(p\) such that \(p = \delta(q,a)\) and \(w\) can take \(p\) to \(q'\) with \(k\) transitions. By
induction hypothesis, \(w\) can also take \(\llangle p \rrangle\) to \(\llangle q'
\rrangle\) with \(k\) transitions in \(M\). Furthermore, \(p =
\delta(q,a)\) implies that \(\llangle p \rrangle = \llangle
\delta(q,a) \rrangle = \delta'(\llangle q \rrangle,a)\); Therefore, \(aw\)
can take \(\llangle q \rrangle\) to \(\llangle q' \rrangle\) with \(k
+ 1\) transitions in \(M\) and thus the statement is true. 
\end{proof}

\begin{lem}
\label{lem:rdfa>mdfa}
\noindent Let a RDFA be \(R = (Q,\ \delta,\ q_0,\ F)\) and its
translated MDFA be \(M = (Q/\!\sim,\ \delta',\ \llangle q_0 \rrangle,\
F')\) according to \autoref{defn:quotient}. For any string \(w\) and
any state \(q\) in \(Q\), \(\delta'^*(\llangle q \rrangle,w) = \llangle \delta^*(q,w) \rrangle\). 
\end{lem}

\begin{proof}
\noindent We can prove the lemma by induction on \(w\). 

\par \noindent \textbf{Base case.}\quad If \(w = \epsilon\),
\(\delta'^*(\llangle q \rrangle,\epsilon) = \llangle q \rrangle =
\llangle \delta^*(q,\epsilon) \rrangle\) and thus the statement is
true. 

\par \noindent \textbf{Induction hypothesis.}\quad For a string \('w\) and
any state \(q\) in \(Q\), \(\delta'^*(\llangle q \rrangle,w') = \llangle \delta^*(q,w') \rrangle\). 

\par \noindent \textbf{Induction step.}\quad If \(w = aw'\), then
\(\delta'^*(\llangle q \rrangle,aw') = \delta'^*(\delta'(\llangle q
\rrangle,a),w')\). Since, \(\delta'(\llangle q \rrangle,a) = \llangle
\delta(q,a) \rrangle\); therefore, \(\delta'^*(\delta'(\llangle q
\rrangle,a),w') = \delta'^*(\llangle
\delta(q,a) \rrangle,w')\). By induction hypothesis,
\(\delta'^*(\llangle \delta(q,a) \rrangle\,w') = \llangle
\delta^*(\delta(q,a),w) \rrangle = \llangle
\delta^*(q,aw) \rrangle\) and thus the statement is true. 
\end{proof}

\par Now, we can prove the correctness of the translation by proving
that their accepting languages are equal. 

\begin{thm}
\noindent For any RDFA, its accepting language is equal to
the language accepted by its translated MDFA, i.e. \(L(\)RDFA\()
= L(\)translated MDFA\()\). 
\end{thm}

\begin{proof}
\noindent Let the RDFA be \(R = (Q,\ \delta,\ q_0,\ F)\) and its
translated MDFA be \(M = (Q/\!\sim,\ \delta',\ \llangle q_0 \rrangle,\
F')\) according to \autoref{defn:quotient}. To prove the theorem, we
have to prove that \(L(R) \subseteq L(M)\) and \(L(R) \supseteq
L(M)\). For any string \(w\),

\par \noindent 1) if \(R\) accepts \(w\), then \(w\) must be able to
take \(q_0\) to an accepting state \(q\) in \(R\). By \autoref{lem:rdfa<mdfa},
\(w\) must be able to take \(\llangle q_0 \rrangle\) to \(\llangle q
\rrangle\) in \(M\). Since \(q \in F\); therefore, \(\llangle q
\rrangle \in F'\) and thus \(M\) accepts \(w\). 

\par \noindent 2) if \(M\) accepts \(w\), then \(w\) must be able to
take \(\llangle q_0 \rrangle\) to an accepting state \(\llangle q
\rrangle\). Then by \autoref{lem:dec_iff}, \(\delta'*(\llangle q_0
\rrangle,w) \in F'\). Furthermore, by \autoref{lem:rdfa>mdfa},
\(\delta'*(\llangle q_0 \rrangle,w) = llangle \delta(q_0,w)
\rrangle)\); therefore, \(llangle \delta(q_0,w) \rrangle \in F'\) and
thus \(delta(q_0,w) \in F\). By \autoref{lem:dec_iff2}, \(w\) can take
\(q_0\) to an accepting state \(q\) in \(R\); therefore, \(R\) accepts
\(w\). 
\end{proof}

\par We also need to prove that the translated MDFA is minimal, but
first, we have to define what is a minimal DFA. 


\subsection{Minimal DFA}
\par The definition of minimal DFA can be found in
\textbf{Correctness/DFA-MDFA.agda}. In order for a DFA to be minimal,
it must satisfy two criteria: 1) every state must be reachable and 2) the
states cannot be further reduced. 

\par Here is the definition of criteria (1) in Agda: 
\begin{lstlisting}[mathescape=true,xleftmargin=.3\textwidth]
All-Reachable-States : DFA $\to$ Set
All-Reachable-States dfa = $\forall$ q $\to$ Reachable q
\end{lstlisting}

\par For criteria 2), we have to first introduce a binary relation of states. 
\begin{defn}
\noindent Suppose we have a DFA, \((Q,\ \delta,\ q_0,\ F)\), for any two states \(p\) and \(q\), we say that a string
\(w\) can distinguish \(p\) and \(q\) if and only if exactly one of
\(\delta^*(p,w)\) and \(\delta^*(q,w)\) is an accepting state,
i.e. \(\delta^*(p,w) \in F \wedge \delta^*(q,w) \notin F \vee
\delta^*(p,w) \notin F \wedge \delta^*(q,w) \in F\). 
\end{defn}

\begin{defn}
\noindent For any two states \(p\) and \(q\) in a given DFA, we say that \(p \nsim
q\) if and only if there exists a string \(w\) that can distinguish
\(p\) and \(q\). 
\end{defn}

\par It is defined in Agda as follow:
\begin{lstlisting}[mathescape=true,xleftmargin=.3\textwidth]
_$\nsim$_ : Q $\to$ Q $\to$ Set
p $\nsim$ q = $\exists$[ w $\in$ $\Sigma^*$ ] 
     ($\delta^*$ p w $\in^d$ F $\times$ $\delta^*$ q w $\notin^d$ F $\uplus$ $\delta^*$ p w $\notin^d$ F $\times$ $\delta^*$ q w $\in^d$ F)
\end{lstlisting}

\begin{defn}
\noindent For a given DFA, it is irreducible if and only if for any
two states \(p\) and \(q\), if \(p\) is not equal to \(q\), then \(p \nsim q\). 
\end{defn}

\par It is defined in Agda as follow:
\begin{lstlisting}[mathescape=true,xleftmargin=.3\textwidth]
Irreducible : DFA $\to$ Set
Irreducible dfa = $\forall$ p q $\to$ $\neg$ p $\approx$ q $\to$ p $\nsim$ q
\end{lstlisting}

\par Now, we can define what is a minimal DFA. 

\begin{defn}
\noindent For a given DFA, it is minimal if and only if all its states
are reachable and it is irreducible. 
\end{defn}

\par It is defined in Agda as follow:
\begin{lstlisting}[mathescape=true,xleftmargin=.3\textwidth]
Minimal : DFA $\to$ Set
Minimal dfa = All-Reachable-States dfa $\times$ Irreducible dfa
\end{lstlisting}

\par Now, let us prove that any MDFA is minimal by proving that all the states in MDFA are reachable and the MDFA
is irreducible. 

\begin{thm}
\label{thm:all_reach}
\noindent For any MDFA, all its states are reachable. 
\end{thm}

\begin{proof}
\noindent Since for any state in a RDFA is reachable, then for any
state \(q\) in RDFA, there must
exist a string \(w\) that can take \(q_0\) to \(q\). By
\autoref{lem:rdfa<mdfa}, \(w\) can also take \(\llangle q_0
\rrangle\) to \(\llangle q \rrangle\) and thus the statement is true. 
\end{proof}

\par To prove that MDFA is irreducible, we have to first prove
that for any two states \(p\) and \(q\), \(\neg (p \sim q\)) if and only if
\(p \nsim q\). 

\begin{lem}
\label{lem:sim_nsim1}
\noindent For any two states \(p\) and \(q\) in a DFA and a string
\(w\), \(\delta^*(p,w) \in F \Leftrightarrow \delta^*(q,w)\) if and
only if \(not exactly one of \delta^*(p,w) and \delta^*(q,w) is
an accepting state\), i.e. \((\delta^*(p,w) \in F \Leftrightarrow
\delta^*(q,w)) \Leftrightarrow \neg (\delta^*(p,w) \in F \wedge \delta^*(q,w) \notin F \vee
\delta^*(p,w) \notin F \wedge \delta^*(q,w) \in F)\). 
\end{lem}

\begin{proof}
\noindent We have to prove for both directions. 

\par \noindent 1) Case \((\delta^*(p,w) \in F \Leftrightarrow
\delta^*(q,w) \in F) \Rightarrow not
exactly one of \delta^*(p,w) and \delta^*(q,w) is
an accepting state\): suppose exactly one of \(\delta^*(p,w)\) and
\(\delta^*(q,w)\) is an accepting state, for example, \(\delta^*(p,w)
\in F \wedge \delta^*(q,w) \notin F\). This contradicts to the
assumption that \(\delta^*(p,w) \in F \Leftrightarrow\). Therefore, not
exactly one of \(\delta^*(p,w)\) and \(\delta^*(q,w)\) is
an accepting state. The same argument also applies for the case when \(\delta^*(p,w)
\notin F \wedge \delta^*(q,w) \in F\).

\par \noindent 2) Case \(not
exactly one of \delta^*(p,w) and \delta^*(q,w) is
an accepting state \Rightarrow (\delta^*(p,w) \in F \Leftrightarrow
\delta^*(q,w) \in F)\): 
\par 2.1) if \(\delta^*(p,w) \in F\), suppose \(\delta^*(q,w) \notin
F\), then this contradicts to the assumption that \(not
exactly one of \delta^*(p,w) and \delta^*(q,w) is
an accepting state\). Therefore, \(\delta^*(q,w) \in F\) is true;
\par 2.2) the same argument applies for the case when \(\delta^*(q,w) \in F\).
\end{proof}


\begin{lem}
\label{lem:sim_nsim}
\noindent For any two states \(p\) and \(q\) in a given DFA, \(\neg (p
\sim q)\) if and only if \(p \nsim q\). 
\end{lem}

\begin{proof}
\noindent We have to prove for both directions.

\par \noindent 1) Case \(\neg (p \sim q) \Rightarrow p \nsim
q\): suppose \(p \nsim q\) is false, then there does not exist a
string \(w\) that can distinguish \(p\) and \(q\). This implies that for any string \(w\), \(w\) cannot
distinguish \(p\) and \(q\), i.e. \(\forall w.\ \neg (\delta^*(p,w) \in F \wedge \delta^*(q,w) \notin F \vee
\delta^*(p,w) \notin F \wedge \delta^*(q,w) \in F)\). By
\autoref{lem:sim_nsim1}, we have \(\forall w.\ \delta^*(p,w) \in F
\Leftrightarrow \delta^*(q,w)\) which is equivalent to \(p \sim
q\). This contradicts to the assumption that \(\neg (p \sim q)\);
therefore, \(p \nsim q\) must be true. 

\par \noindent 2) Case \(p \nsim q \Rightarrow \neg (p \sim q)\): suppose \(p \sim q\), then for any string \(w\),
\(\delta^*(p,w) \in F \Leftrightarrow \delta^*(q,w)\). Since \(p \nsim
q\), then there must exist a string \(w\)
that can distinguish \(p\) and \(q\), i.e. exactly one of
\(\delta^*(p,w)\) and \(\delta^*(q,w)\) is an accepting state. This
contradicts to the assumption that \(\delta^*(p,w) \in F
\Leftrightarrow \delta^*(q,w)\), therefore, \(p \sim q\) must be true. 
\end{proof}

\begin{thm}
\label{thm:mdfa_irreducible}
\noindent For any MDFA, it is irreducible, i.e. for any two states
\(p\) and \(q\) in the MDFA, if \(p\) is not equal to \(q\) then \(p \nsim q\). 
\end{thm}

\begin{proof}
\noindent For any two states \(\llangle p \rrangle\) and \(\llangle q
\rrangle\) in MDFA, if \(\llangle p \rrangle\) is not equal to \(\llangle q
\rrangle\), then \(p \sim q\) is false. By \autoref{lem:sim_nsim},
\(p \nsim q\) is true. 
\end{proof}

\begin{thm}
\noindent For any MDFA, it is minimal, i.e. all the states in the MDFA
are reachable and the MDFA is irreducible. 
\end{thm}

\begin{proof}
\noindent The statement follows from \autoref{thm:all_reach}
and \autoref{thm:mdfa_irreducible}. 
\end{proof}


