\subsection{Non-deterministic Finite Automata}
\par Although the definition of NFA is very similar to that of
\(\epsilon\)-NFA, we will still define NFA
separately. The type of NFA is defined in \textbf{NFA.agda} along with
its operations and properties. 

\begin{defn}
\noindent A NFA is a 5-tuple \(M = (Q
,\ \Sigma,\ \delta,\ q_0,\ F)\), where
\begin{enumerate}[nolistsep]
  \item \(Q\) is a finite set of states;
  \item \(\Sigma\) is the set of alphabets;
  \item \(\delta\) is a mapping from \(Q \times \Sigma\) to
    \(\mathcal P \left({Q}\right)\) that defines the behaviour of the automata;
  \item \(q_0\) in \(Q\) is the initial state;
  \item \(F \subseteq Q\) is the set of accepting states. 
\end{enumerate}
\end{defn}

\par It is formalised as a record in Agda as shown below: 

\begin{lstlisting}[mathescape=true,xleftmargin=.3\textwidth]
record NFA : Set$_1$ where
  field
    Q      : Set
    $\delta$       : Q $\to$ $\Sigma$ $\to$ DecSubset Q
    q$_0$      : Q
    F      : DecSubset Q
    Q?     : DecEq Q
    |Q|-1  : $\mathbb{N}$
    It     : Vec Q (suc |Q|-1)
    $\forall$q$\in$It    : (q : Q) $\to$ (q $\in^V$ It)
    unique : Unique It
\end{lstlisting}

\par The set of alphabets \(\Sigma\) is passed into the module as a
parameter. Together with \(Q\), \(\delta\),
\(q_0\) and \(F\), these five fields correspond to the 5-tuple
NFA. The other extra fields are used in powerset construction. They
are \(Q?\) -- the decidable equality of \(Q\);
\(|Q|-1\) -- the number of states minus 1; \(It\) -- a vector containing every state in \(Q\); \(\forall q\in It\)
-- a proof that every state in \(Q\) is also in the vector
\(It\); and \(unique\) -- a proof that there is no repeating elements in
\(It\). 

\par Now, before we can define the accepting language of a given
NFA, we need to define several operations of NFA. 

\begin{defn}
\noindent A configuration is composed of a state and an alphabet from
\(\Sigma\), i.e. \(C = Q \times \Sigma\). 
\end{defn}

\begin{defn}
\noindent A move in an NFA is
represented by a binary function (\(\vdash\)) on two configurations. We say
that for all \(w \in \Sigma^*\) and \(a \in \Sigma\), \((q, aw)
\vdash (q' , w)\) if and only if \(q' \in \delta (q , a)\). 
\end{defn}

\par The binary function is defined in Agda as follow: 
\begin{lstlisting}[mathescape=true,xleftmargin=.3\textwidth]
  _$\vdash$_ : (Q $\times$ $\Sigma$ $\times$ $\Sigma^*$) $\to$ (Q $\times$ $\Sigma^*$) $\to$ Set
  (q , a , w) $\vdash$ (q' , w') = w $\equiv$ w' $\times$ q' $\in^d$ $\delta$ q a
\end{lstlisting}

\begin{defn}
\noindent Suppose \(C\) and \(C'\) are configurations. We say that \(C \vdash^0 C'\) if and only
if \(C = C'\); and \(C_0 \vdash^k C_k\) for any \(k \geq 1\) if and only if there exists a chain of
configurations \(C_1, C_2, ..., C_{k-1}\) such that \(C_i \vdash C_{i+1}\) for all \(0 \leq i < k\). 
\end{defn}

\par It is defined as a recursive function in Agda as follow: 
\begin{lstlisting}[mathescape=true,xleftmargin=.3\textwidth]
  _$\vdash^k$_-_ : (Q $\times$ $\Sigma^*$) $\to$ $\mathbb{N}$ $\to$ (Q $\times$ $\Sigma^*$) $\to$ Set
  (q , w) $\vdash^k$ zero  - (q' , w')
    = q $\equiv$ q' $\times$ w $\equiv$ w'
  (q , w) $\vdash^k$ suc n - (q' , w') 
    = $\exists$[ p $\in$ Q ] $\exists$[ a $\in$ $\Sigma$ ] $\exists$[ u $\in$ $\Sigma^*$ ]
      (w $\equiv$ a :: u $\times$ (q , a , u) $\vdash$ (p , u) $\times$ (p , u) $\vdash^k$ n - (q' , w'))
\end{lstlisting}

\begin{defn}
\noindent We say that \(C \vdash^* C'\) if and only
if there exists a number of chains \(n\) such that \(C \vdash^n C'\). 
\end{defn}

\par Its corresponding type is defined as follow: 
\begin{lstlisting}[mathescape=true,xleftmargin=.3\textwidth]
  _$\vdash^*$_ : (Q $\times$ $\Sigma^*$) $\to$ (Q $\times$ $\Sigma^*$) $\to$ Set
  (q , w) $\vdash^*$ (q' , w') = $\exists$[ n $\in$ $\mathbb{N}$ ] (q,w) $\vdash^k$ n - (q',w')
\end{lstlisting}

\begin{defn}
\noindent For any string \(w\), it is accepted by an NFA
if and only \(w\) can take \(q_0\) to an accepting state \(q\). Therefore, the
accepting language of an NFA is given by the set \(\{w\ |\ \exists q\in F.\ (q_0,w) \vdash^* (q,\epsilon)\}\). 
\end{defn}

\par The corresponding formalisation in Agda is as follow: 
\begin{lstlisting}[mathescape=true,xleftmargin=.3\textwidth]
  L$^N$ : NFA $\to$ Language
  L$^N$ nfa = $\lambda$ w $\to$ 
            $\exists$[ q $\in$ Q ] (q $\in^d$ F $\times$ (q$_0$ , w) $\vdash^*$ (q , [])))
\end{lstlisting} 


\subsection{Removing \(\epsilon\)-transitions}
\par The translation of \(\epsilon\)-NFA to NFA is defined in
\textbf{Translation/eNFA-NFA.agda}. In order to remove all the \(\epsilon\)-transitions, we need to
know whether a state \(q\) can reach another state \(q'\) with only
\(\epsilon\)-transitions. Let us begin by defining such a relation on states. 

\begin{defn}
\noindent We say that
\(q \to_\epsilon^0 q'\) if and only if
\(q\) is equal to \(q'\); and \(q \to_\epsilon^k q'\) for \(k \geq
1\) if and only if \(q\) can be transited to \(q'\) with \(k\)
\(\epsilon\)-transitions. We call this an \(\epsilon\)-path from \(q\) to \(q'\).
\end{defn}

\par It is defined as a recursive function in Agda as follow:
\begin{lstlisting}[mathescape=true,xleftmargin=.3\textwidth]
_$\to_\epsilon^k$_-_ : Q $\to$ $\mathbb{N}$ $\to$ Q $\to$ Set
q $\to_\epsilon^k$ zero - q' = q $\equiv$ q'
q $\to_\epsilon^k$ suc n - q' = $\exists$[ p $\in$ Q ] ( p $\in^d$ $\delta$ q E $\times$ p $\to_\epsilon^k$ n - q' )
\end{lstlisting}

\begin{defn}
\noindent We say that \(q \to_\epsilon^* q'\) if and only if there
exists an \(\epsilon\)-path from \(q\) to \(q'\) with \(n\) transitions, i.e. \(\exists n.\ q \to_\epsilon^n q'\). 
\end{defn}

\par The corresponding type is as follow: 
\begin{lstlisting}[mathescape=true,xleftmargin=.3\textwidth]
_$\to_\epsilon^*$_ : Q $\to$ Q $\to$ Set
q $\to_\epsilon^*$ q' = $\exists$[ n $\in$ $\mathbb{N}$ ] q $\to_\epsilon^k$ n - q'
\end{lstlisting}

\par Now we have to prove that for any two states \(q\) and \(q'\), 
\(q \to_\epsilon^* q'\) is decidable. However, it is not possible to
prove it directly because the set of natural numbers is
infinite. Therefore, we will introduce an algorithm that computes the
\(\epsilon\)-closure of a state. The \(\epsilon\)-closure of a state
\(q\), \(\epsilon\hyphen closure(q)\) should contain all the states
that are reachable from \(q\) with only \(\epsilon\)-transitions. We will prove that for any two states
\(q\) and \(q'\), \(q \to_\epsilon^* q'\) if and only if \(q' \in \epsilon\hyphen closure(q)\). By proving that they are equivalent, we
will have proved the decidability of \(q \to_\epsilon^* q'\). 

\begin{defn}
\noindent For any given state \(q\), \(\epsilon\hyphen closure(q)\) is obtained by: 
\begin{enumerate}[nolistsep]
  \item put \(q\) into a subset \(S\), i.e. \(S = \{q\}\)
  \item loop for \(|Q| - 1\) times: 
  \begin{enumerate}
    \item for every state \(p\) in \(S\), if \(\epsilon\) can take
      \(p\) to another state \(r\), i.e. \(r \in
        \delta (p,\epsilon)\), then put \(r\) into \(S\). 
  \end{enumerate}
  \item the result subset \(S\) is the \(\epsilon\)-closure of \(q\)
\end{enumerate}
\end{defn}

\begin{lem}
\noindent For any two states \(q\) and \(q'\), \(q \to_\epsilon^* q'\)
if and only if \(q' \in \epsilon\hyphen closure(q)\). 
\end{lem}

\begin{proof}
\noindent We have to prove for both directions. 
\par \noindent 1) If \(q \to_\epsilon^* q'\), then there must exist a
number \(n\) such that \(\epsilon^n\) can take \(q\) to \(q'\). If
\(n < |Q|\), then it is obvious that \(q' \in
\epsilon\hyphen closure(q)\) is true. If \(n \geq |Q|\), the
\(\epsilon\)-path from \(q\) to \(q'\) must have loop(s) inside. By
removing the loop(s), the equivalent \(\epsilon\)-path must have at
most \(|Q|-1\) \(\epsilon\)-transitions. Therefore, \(q' \in
\epsilon\hyphen closure(q)\) is true. 

\par \noindent 2) If \(q' \in \epsilon\hyphen closure(q)\), it is obvious
that \(q \to_\epsilon^{|Q|-1} q'\) must be true and thus \(q
\to_\epsilon^* q'\) is true. 
\end{proof}

\par Since we have proved that they are equivalent; therefore the
decidability of \(q \to_\epsilon^* q'\) follows. Now, let us define
the translation of \(\epsilon\)-NFA to NFA using \(q \to_\epsilon^* q'\). 
 
\begin{defn}
\label{defn:remove_epsilon}
\noindent For a given \(\epsilon\)-NFA, \((Q,\ \delta,\
q_0,\ F)\), its translated NFA will be \((Q,\ \delta',\ q_0,
F')\) where
\begin{itemize}[nolistsep]
  \item \(\delta'(q,a) = \delta (q,a) \cup \{q'\ |\ \exists p.\
      q \to_\epsilon^* p \wedge q' \in \delta (p,a)\}\);
  \item \(F' = F \cup \{q\ |\ \exists p\in F.\ q \to_\epsilon^* p\}
    \). 
\end{itemize}
\end{defn}

\par Now, let us prove the correctness of the above translation by proving their accepting languages
are equal. The correctness proofs can be found in \textbf{Correctness/eNFA-NFA.agda}.

\begin{thm}
\noindent For any \(\epsilon\)-NFA, its accepting language is equal to
the language accepted by its translated NFA, i.e. \(L(\epsilon\)-NFA\()
= L(\)translated NFA\()\). 
\end{thm}

\begin{proof}
\noindent Let the \(\epsilon\)-NFA be \(\epsilon N = (Q,\ \delta,\ q_0,\
F)\), and its translated NFA be \(N = (Q,\ \delta',\ q_0,\ F')\)
according to \autoref{defn:remove_epsilon}. To
prove the theorem, we have to prove that \(L(\epsilon N) \subseteq
L(N)\) and \(L(\epsilon N) \supseteq L(N)\). Then for any string \(w\), 

\par \noindent 1) if \(\epsilon N\) accepts \(w\), then there must exist an
\(\epsilon\)-string of \(w\), \(w^e\), that can take \(q_0\) to an
accepting state \(q\) with \(n\) transitions. There are three
possibilities: a) the last transition in the path is not an
\(\epsilon\)-transition, b) the path is divided into three parts, the
first part from \(q_0\) to a state \(p\) with less than \(n\)
transitions; the second part from \(p\) to a state \(p_1\) with an
alphabet \(a\) and the third part from \(p_1\) to \(q\) with only
\(\epsilon\)-transitions and c) the path from \(q_0\)
to \(q\) consists of only \(\epsilon\)-transitions.

\par \textit{Case a}:\quad There must exist a state \(p\), an
\(\epsilon\)-string \(u^e\), and an alphabet \(a\) such that \(u^e\)
can take \(q_0\) to \(p\) in less than \(n\) transitions, \(q \in
\delta(p,a)\) and \(w^e = u^ea\). Since that path from \(q_0\) to
\(p\) is less than \(n\) transitions; therefore, we can prove by
induction that the normal string of \(u^e\), \(u\), can take
\(q_0\) to \(p\) in \(N\). Furthermore, \(w = ua\); therefore, \(w\) can
take \(q_0\) to \(q\) in \(N\) and thus \(N\) accepts \(w\). 

\par \textit{Case b}:\quad There must exist two states \(p\) and \(p_1\), an
\(\epsilon\)-string of \(w\), \(u^e\), and an alphabet \(a\) such
that \(u^e\) can take \(q_0\) to \(p\) in less than \(n\) transitions,
\(p_1 \in \delta(p,a)\), \(p_1 \to_\epsilon^* q\) and \(w = ua\). Since that path from \(q_0\) to
\(p\) is less than \(n\) transitions; therefore, we can prove by
induction that the normal string of \(u^e\), \(u\), can take
\(q_0\) to \(p\) in \(N\). Furthermore, \(p_1\) must be an accepting state in \(N\)
because it can be transited to an accepting state \(q\) with only
\(\epsilon\)-transitions. Therefore, \(w\) can take \(q_0\) to an
accepting state \(p_1\) in \(N\) and thus \(N\) accepts \(w\). 

\par \textit{Case c}:\quad If the path from \(q_0\)
to \(q\) consists of only \(\epsilon\)-transitions, then \(q_0
\to_\epsilon^* q\) is true; therefore, \(q_0\) is also an accepting
state in \(N\). Furthermore, the accepted string consists of \(\epsilon\) only;
therefore, \(w = \epsilon\). It is obvious that \(N\) accepts \(\epsilon\). 

\par \noindent 2) if \(N\) accepts \(w\), then \(w\) must be able to take \(q_0\) to an
accepting state \(q\). Since \(q\) is an accepting state, then \(q\)
is also an accepting state in \(\epsilon N\) or there exists a state 
\(p\) such that \(q \to_\epsilon^* p\) and \(p \in F\). For the former
case, since \(w\) is also an \(\epsilon\)-string of itself; therefore,
it is obvious that \(\epsilon N\) accepts \(w\). For the latter
case, suppose the path from \(q\) to \(p\) has \(n\)
\(\epsilon\)-transitions, the consider another \(\epsilon\)-string of
\(w\), \(w\epsilon^n\), it can take \(q_0\) to \(p\) in
\(\epsilon N\). Therefore \(\epsilon N\) accepts \(w\). 
\end{proof}