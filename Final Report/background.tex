\section{Background}

\subsection{Agda} 
\paragraph{} Agda is a dependently-typed functional programming language and a
proof assistant based on Intuitionistic Type Theory
\cite{martin1984}. The current version (Agda 2) is rewritten by Norell
\cite{norell2007} during his study at the Chalmers University of
Technology. In this section, we will describe the basic features of
Agda and how dependent types are employed to construct programs and
proofs. Most of the materials presented below can also be
found in the two tutorial papers \cite{bove2009} and
\cite{norell2009}. Interested readers can take a look in order to get
a more precise idea on how to work with Agda. Now, let us begin by
showing how to do ordinary functional programming in Agda. 


\subsubsection{Simply Typed Functional Programming}
\paragraph{} Haskell is the implementaion language
of Agda, as we will see below, Agda has borrowed many features from
Haskell. In the following paragraphs, we will demonstrate how to
define basic data types and functions. 

\paragraph{Boolean} We first define the type of boolean values in Agda.  
\begin{lstlisting}[mathescape=true,xleftmargin=.3\textwidth]
data Bool : Set where
  true  : Bool
  false : Bool
\end{lstlisting}

\paragraph{} \(Bool\) is a data type which has two
constructors: \(true\) and \(false\). Note that these two constructors
are also elements of \(Bool\) as they take no arguments. On
the other hand, \(Bool\) is a member of the type
\(Set\). The type of \(Set\) is \(Set_1\) and the type of \(Set_1\) is
\(Set_2\). The type hierarchy goes on and becomes infinite. Now, let
us define the negation function over boolean values. 
\begin{lstlisting}[mathescape=true,xleftmargin=.3\textwidth]
not : Bool $\to$ Bool
not true  = false
not false = true
\end{lstlisting}

\paragraph{} In Agda, we must explicitly provide a type declaration for every
variable. Note that we can declare partial
functions in Haskell but not in Agda. For instance, the function below
will be rejected by the Agda compiler.
\begin{lstlisting}[mathescape=true,xleftmargin=.3\textwidth]
not : Bool $\to$ Bool
not true  = false
\end{lstlisting}

\paragraph{Natural Number} Now, let us define the type of natural numbers
inductively as follow: 
\begin{lstlisting}[mathescape=true,xleftmargin=.3\textwidth]
data $\mathbb N$ : Set where
  zero : $\mathbb N$
  suc  : $\mathbb N$ $\to$ $\mathbb N$
\end{lstlisting} 

\paragraph{} The constructor \(suc\) represents the successor of a given
natural number. For instance, \(1\) is represented by \(suc\
zero\). Now, let us define the addition function of natural numbers recursively as follow:
\begin{lstlisting}[mathescape=true,xleftmargin=.3\textwidth]
_+_ : $\mathbb N$ $\to$ $\mathbb N$ $\to$ $\mathbb N$: Set where
zero + m = m
(suc n) + m = suc (n + m)
\end{lstlisting} 

\paragraph{Parameterised Types} In Haskell, the type of list \([a]\) is parameterised by the type
parameter \(a\). The analogus data type in Agda is defined as follow:
\begin{lstlisting}[mathescape=true,xleftmargin=.3\textwidth]
data List (A : Set) : Set where
  []   : List A
  _::_ : A $\to$ List A $\to$ List A
\end{lstlisting} 

\paragraph{} Let us try to define the function \(head\) which will
return the first element of a given list. 
\begin{lstlisting}[mathescape=true,xleftmargin=.3\textwidth]
head : {A : Set} $\to$ List A $\to$ A
head [] = ?
head (x :: xs) = x
\end{lstlisting} 

\paragraph{} In Haskell, we can simply skip the case
\([\ ]\) and it will produce an error for us. However, as we
mentioned before, we have to pattern match every case in Agda. One
possible workaround is to use the \(Maybe\) type. However, we can do a
better job with dependent types. 


\subsubsection{Dependent Types}
\paragraph{} Dependent types are types that depend on values of other
types. For example, \(A^n\) is a vector that contains \(n\) elements
of \(A\). This type is not possible
to be declared in simply-typed systems like Haskell or Ocaml. Now, let
us look at how it is defined in Agda using dependent type. 
\begin{lstlisting}[mathescape=true,xleftmargin=.25\textwidth]
data Vec (A : Set) : $\mathbb N$ $\to$ Set where
  []   : Vec A zero
  _::_ : $\forall$ {n} $\to$ A $\to$ Vec A n $\to$ Vec A (suc n)
\end{lstlisting} 

\paragraph{} In the type declaration \textbf{\(data\ Vec\ (A : Set) :\ \mathbb N \to
Set\)}, \(A : Set\) is the type parameter defining the type of
elements that will be in the vector. While the part \(\mathbb N \to Set\) means
that \(Vec\) takes a number \(n\) from the set \(\mathbb N\) and produces a
type that depends on \(n\). With different natural number
\(n\), we get different type from the inductive family \(Vec\). For example, \(Vec\ A\ zero\) is
the type of empty vectors and \(Vec\ A\ 10\) is another vector type with length ten. 


\paragraph{} Dependent types allow us to be more
expressive and precise over type declaration. Let us declare the
\(head\) function for \(Vec\). 
\begin{lstlisting}[mathescape=true,xleftmargin=.25\textwidth]
head : {A : Set}{n : $\mathbb N$} $\to$ Vec A (suc n) $\to$ A
head (x :: xs) = x 
\end{lstlisting} 

\paragraph{} We only need to pattern match on the case (\(x
:: xs\)) because the type \(Vec\ A\ (suc\ n)\) ensures that the argument will never
be \([\ ]\). Apart from vectors, we can also define the type of binary
search tree in which every tree is guranteed to be
sorted. However, we will not be looking into that in here as it
is not our major concern. Interersted readers can take a look at 
Section 6 in \cite{bove2009}. Furthermore, dependent types also allow
us to encode predicate logic and program specifications as
types. Before we go into these two applications, we will first
introduce the idea of propositions as types. 


\subsubsection{Propositions as Types}
\paragraph{} In the 1930s, Curry identified the
correspondence between propositions in propositional logic and types
\cite{curry1934}. After that, in the 1960s, de Bruijn and Howard extended
Curry's correspondence to predicate logic by introducing dependent
types \cite{bruijn1968, howard1969}. Later on, Martin-L\"of published
his work, Intuitionistic Type Theory \cite{martin1984}, which turned the correspondence into a new
foundational system for constructive mathematics. 

\paragraph{} In the paragraphs below, we will show how the correspondence is
formalised in Agda. Note that the Intuitionistic Type
Theory is based on constructive logic but not classical logic. There
is fundamental difference between them. Interested readers can take a look at
\cite{avigad2000}. Now, we will first begin with the formalisation of
propositional logic. 

\paragraph{Propositional Logic} In general, Curry's correspondence
states that a proposition can be interpreted as a set of its proofs. A
proposition is true if and only if its set of proofs is inhabited,
i.e. there is at least one element in the set; it is false if and only
if its set of proofs is empty. 

\subparagraph{Truth} For a proposition always to be true, the corresponding
type must have at least one element. 
\begin{lstlisting}[mathescape=true,xleftmargin=.3\textwidth]
data $\top$ : Set where
  tt : $\top$
\end{lstlisting} 

\subparagraph{Falsehood} For a proposition always to be false, the corresponding
type must have no elements at all. 
\begin{lstlisting}[mathescape=true,xleftmargin=.3\textwidth]
data $\bot$ : Set where
\end{lstlisting} 

\subparagraph{Conjunction} Suppose \(A\) and \(B\) are propositions, then the
proofs of \(A \wedge B\) should be consist of
a proof of \(A\) and a proof of \(B\). In Type Theory, it corresponds
to the product type. 
\begin{lstlisting}[mathescape=true,xleftmargin=.3\textwidth]
data _$\times$_ (A B : Set) : Set where
  _,_ : A $\to$ B $\to$ A $\times$ B
\end{lstlisting} 

\paragraph{} The above construction resembles the introduction rule of
conjunction while the elimination rules are formalised as follow:
\begin{lstlisting}[mathescape=true,xleftmargin=.3\textwidth]
fst : {A B : Set} $\to$ A $\times$ B $\to$ A
fst (a , b) = a

snd : {A B : Set} $\to$ A $\times$ B $\to$ B
snd (a , b) = b
\end{lstlisting} 


\subparagraph{Disjunction} Suppose \(A\) and \(B\) are propositions, then the
proofs of \(A \vee B\) should contains either a proof of \(A\) or a
proof of \(B\). In Type Theory, it is represented by the sum type. 
\begin{lstlisting}[mathescape=true,xleftmargin=.3\textwidth]
data _$\uplus$_ (A B : Set) : Set where
  inj$_1$ : A $\to$ A $\uplus$ B
  inj$_2$ : B $\to$ A $\uplus$ B
\end{lstlisting} 

\paragraph{} Now, we can define the elimination rule of disjunction as
follow: 
\begin{lstlisting}[mathescape=true,xleftmargin=.2\textwidth]
$\uplus \hyphen$elim : {A B C : Set} $\to$ A $\uplus$ B $\to$ (A $\to$ C) $\to$ (B $\to$ C) $\to$ C
$\uplus \hyphen$elim (inj$_1$ a) f g = f a
$\uplus \hyphen$elim (inj$_2$ b) f g = g b
\end{lstlisting} 


\subparagraph{Negation} Suppose \(A\) is a proposition, then its negation is
defined as a function that transform every arbitary proof in \(A\) to
the falsehood (\(\bot\)). 
\begin{lstlisting}[mathescape=true,xleftmargin=.3\textwidth]
$\neg$ : Set $\to$ Set
$\neg$ A = A $\to$ $\bot$
\end{lstlisting} 


\subparagraph{Implication} We say that \(A\) implies \(B\) if and only if we can
transform every proof of \(A\) into a proof of \(B\). In Type
Theory, it corresponds to a function from \(A\) to \(B\), i.e. \(A \to
B\). 

\subparagraph{Equivalence} Equivalence of two propositions \(A\) and
\(B\) can be seperated into \(A\) implies \(B\) and \(B\) implies
\(A\). Therefore, we can consider it as a conjunction of the two
implications.
\begin{lstlisting}[mathescape=true,xleftmargin=.3\textwidth]
_$\iff$_ : Set $\to$ Set $\to$ Set
A $\iff$ B = (A $\to$ B) $\times$ (B $\to$ A)
\end{lstlisting} 


\paragraph{} Now, we can prove some theorems in propositional logic
using the formalision above, for example, if \(P\) implies \(Q\) and
\(Q\) implies \(R\), then \(P\) implies \(R\). 
\begin{lstlisting}[mathescape=true,xleftmargin=.3\textwidth]
prop-lem : {P Q : Set} $\to$ (P $\to$ Q) $\to$ (Q $\to$ R) $\to$ (P $\to$ R)
prop-lem f g = $\lambda$ p $\to$ g (f p)
\end{lstlisting} 

\paragraph{} By completing the function, we have provided an element
to the type \((P \to Q) \to (Q \to R) \to (P \to R)\). Therefore, we have
also proved that the proposition is true. 


\paragraph{Predicate Logic} We will now move on to predicate logic and
to introduce the universal (\(\forall\)) and existenial (\(\exists\))
quantifiers. A predicate is represented by a dependent type in the
form of \(A \to Set\). For example, we can
define the predicate of even numbers and odd numbers inductively as follow:
\begin{lstlisting}[mathescape=true,xleftmargin=.25\textwidth]
data _isEven : $\mathbb N$ $\to$ Set where
  base : zero isEven
  step : $\forall$ n $\to$ n isEven $\to$ (suc (suc n)) isEven

data _isOdd : $\mathbb N$ $\to$ Set where
  base : (suc zero) isOdd
  step : $\forall$ n $\to$ n isOdd $\to$ (suc (suc n)) isOdd
\end{lstlisting} 


\subparagraph{Universal Quantifier} The interpretaion of the universal quantifier is similar to
implication. In order for \(\forall x\in A. B(x)\) to be true, we will have
to transform every proof \(a\) of \(A\) into a proof of the predicate
\(B[x:=a]\). In Type Theory, it is represented by the function \((x :
A) \to B\ x\). For example, we can prove by induction that for every natural
number, it is either even or odd.
\begin{lstlisting}[mathescape=true,xleftmargin=.25\textwidth]
lem$_1$ : $\forall$ n $\to$ n isEven $\uplus$ n isOdd
lem$_1$ zero = inj$_1$ base
lem$_1$ (suc zero) = inj$_2$ base
lem$_1$ (suc (suc n)) with lem$_1$ n
... | inj$_1$ nIsEven = inj$_1$ (step n nIsEven)
... | inj$_2$ nIsOdd = inj$_2$ (step n nIsOdd)
\end{lstlisting} 

\subparagraph{Existential Quantifier} The interpretation of the
existensial quantifier is simliar to conjunction. In order for
\(\exists x\in A. B(x)\) to be true, we need to provide a proof
\(a\) of the \(A\) and a proof \(p\) of the predicate
\(B[x:=a]\). In Type Theory, it is represented by the generalised
product type \(\Sigma\). 
\begin{lstlisting}[mathescape=true,xleftmargin=.25\textwidth]
data $\Sigma$ (A : Set) (B : A $\to$ Set) : where
  _,_ : (a : A) $\to$ B a $\to$ $\Sigma$ A B
\end{lstlisting}
For simplicity, we will change the syntax of \(\Sigma\) to \(\exists [ x \in A ] B\). Now, we can prove that
there exists a natural number which is even. 
\begin{lstlisting}[mathescape=true,xleftmargin=.25\textwidth]
lem$_2$ : $\exists$[ n $\to$ $\mathbb N$ ] n isEven
lem$_2$ = zero , base
\end{lstlisting}


\paragraph{Decidability} A proposition \(P\) is decidable if and only if there
exists an algorithm that can decide whether it is true or false. We
will define the decidability of a proposition as follow:
\begin{lstlisting}[mathescape=true,xleftmargin=.25\textwidth]
data Dec (A : Set) : Set where
  yes : A $\to$ Dec A
  no  : $\neg$ A $\to$ Dec A
\end{lstlisting}


\paragraph{Propositional Equality} Another important feature of Type Theory is
that we can define the notion of equality in it. The
equality relation is interpreted as follow:
\begin{lstlisting}[mathescape=true,xleftmargin=.25\textwidth]
data _$\equiv$_ {A : Set} (x : A) : A $\to$ Set where
  refl : x $\equiv$ x
\end{lstlisting}
\paragraph{} This states that for any \(x\) in \(A\), \(refl\) is an
element of the type \(x \equiv x\). More generally, \(refl\) is a
proof of \(x \equiv x'\) provided that \(x\) and \(x'\) is the same
after normalisation. For example, we can prove that \(\exists n\in
\mathbb N . n = 1 + 1\) as follow:
\begin{lstlisting}[mathescape=true,xleftmargin=.25\textwidth]
lem$_3$ : $\exists$[ n $\in$ $\mathbb N$ ] n $\equiv$ (1 + 1)
lem$_3$ = suc (suc zero) , refl
\end{lstlisting}

\paragraph{} Note that we can put \(refl\) in the proof only because both
\(suc (suc zero)\) and \(1 + 1\) are the same after
normalisation. Now, let us define the elimination rule of
equality. The rule should allow us to substitute equivalence objects
into any proposition. 
\begin{lstlisting}[mathescape=true,xleftmargin=.25\textwidth]
subst : {A : Set}{x y : A} $\to$ (P : A $\to$ Set) $\to$ x $\equiv$ y $\to$ P x $\to$ P y
subst P refl p = p 
\end{lstlisting}
\paragraph{} We can also prove the congruency of equality: 
\begin{lstlisting}[mathescape=true,xleftmargin=.25\textwidth]
cong : {A B : Set}{x y : A} $\to$ (f : A $\to$ B) $\to$ x $\equiv$ y $\to$ f x $\equiv$ f y
cong f refl = refl
\end{lstlisting}


\subsubsection{Program Specifications as Types}
\paragraph{} Furthermore, dependent types also allow us to encode program
specification within the same platform. Now, let us define a predicate
of sorted list (sorted by ascending order). For simplicity, we will only consider the list of
natural numbers in here. Before we can define the predicate, we need
to have another predicate saying that a given natural number is
smaller than every number in a given list. 
\begin{lstlisting}[mathescape=true,xleftmargin=.25\textwidth]
All$\hyphen$lt : $\mathbb N$ $\to$ List $\mathbb N$ $\to$ Set
All$\hyphen$lt n [] = $\top$
All$\hyphen$lt n (x :: xs) = n $\leq$ x $\times$ All$\hyphen$lt n xs

Sorted$\hyphen$ASC : List $\mathbb N$ $\to$ Set
Sorted$\hyphen$ASC [] = $\top$
Sorted$\hyphen$ASC (x :: xs) = All$\hyphen$lt x xs $\times$ Sorted$\hyphen$ASC xs
\end{lstlisting}

\paragraph{} \(\_\leq\_\ :\ \mathbb N \to \mathbb N \to Set\) is a binary relation between two natural
numbers. Now, we can define an insertion function that takes a
natural number and a list as the arguments and returns a list of
natural numners. The insertion function is designed so that if the
input list is already sorted, then the output list should also be sorted. 

\begin{lstlisting}[mathescape=true,xleftmargin=.25\textwidth]
insert : $\mathbb N$ $\to$ List $\mathbb N$ $\to$ List $\mathbb N$
insert n [] = n :: []
insert n (x :: xs) with n $\leq$? x
... | yes _ = n :: (x :: xs)
... | no  _ = x :: insert n xs
\end{lstlisting}

\paragraph{} \(\_\leq ?\_\ :\ \forall n\ m \to Dec (n \leq m)\) is the
decidability of \(\_\leq\_\), it can also be used to determine whether
a given number \(n\) is less than or equal to another number
\(m\). Now, let us encode the specification of the insertion function
as follow: 
\begin{lstlisting}[mathescape=true,xleftmargin=.25\textwidth]
insert$\hyphen$sorted : $\forall$ {n} {as} 
                $\to$ Sorted$\hyphen$ASC as 
                $\to$ Sorted$\hyphen$ASC (insert n as)
\end{lstlisting}

\paragraph{} The part \(Sorted\hyphen ASC\ as\) corresponds to the pre-condition and
\(Sorted\hyphen ASC\ (insert\ n\ as\) corresponds to the
post-condition. Once we have completed the
function, we will also have proven the specification to be
true. Interested readers can try to finish the proof. 


\subsection{Related Work}
\paragraph{} Firsov and Uustalu \cite{firsov2013} have conducted a
similar research on the formalisation of Automata Theory. In their
work, they have formalised the translation of regular expressions to NFA together with
its correctness proofs in Agda. In their definition
of NFA, the set of states and its subsets were represented as
vectors. ... 