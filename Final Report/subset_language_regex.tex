\section{Subsets, Decidable Subsets and Vector Representation}
\par The types of subsets and decidable subsets are defined in
\textbf{Subset.agda} and \textbf{Subset/DecidableSubset.agda}
respectively along with their operations such as membership (\(\in\)), subset
(\(\subseteq\)), superset (\(\supseteq\)) and equality (\(=\)). To separate the
operations of subsets and decidable subsets, all the operations of
decidable subset are denoted by the superscript (\(^d\)), e.g. \(\in^d\)
is the membership decider for decidable subsets. Let us
begin with the definition of general subsets. 

\begin{defn} 
\noindent Suppose \(A\) is a set, then its
subsets are represented as unary functions on
\(A\) in Type Theory, i.e. \(Subset\ A = A \to Set\). 
\end{defn}

\par In our definition, a subset is a function from \(A\) to
\(Set\). When declaring a subset, we can write \(sub =
\lambda\ (x : A) \to P\ x\). \(P\ x\) defines the conditions for \(x\) to
be included in \(sub\). This construction is
very similar to set comprehension. For example, the above subset
resembles the set \(\{x\ | \ x \in A,\ P(x)\}\). Furthermore, \(sub\) is
also a predicate on \(A\) as its type is in the form of \(A \to
Set\) and its decidability will remain unknown until it is either proved or disproved. 

\begin{defn} 
\noindent Another representation of subsets is \(DecSubset\ A = A \to
Bool\). Unlike \(Subset\), its decidability is ensured by its
definition. 
\end{defn}

\par The two representations have different roles in other parts of the
formalisaton. \(Language\) is defined using \(Subset\) as not every
language is decidable. For other parts in the project 
such as the subsets of states in an automaton, \(DecSubset\) is used
because the decidability is assumed. 


\subsection{Vector Representation}
\par Recall that Firsov and Uustalu \cite{firsov2013} represent the
set of states and its subsets as vectors. However, this makes the definition of
NFA becomes unnatural. Therefore, at the beginning, we intended to
avoid the vector representation. However, it is impossible because we have to iterate the subset
when computing \(\epsilon\)-closures. The problems will be discussed in
Chapter 6 in detail. The vector representation is defined in
\textbf{Subset/VectorRep.agda} along with its operations and proofs. 

\par In order to use the vector representation, several objects must
be passed to the module \textbf{Vec-Rep}. They are \((A : Set)\) -- a finite set;
\((dec : DecEq\ A)\) -- the decidable equality of \(A\); \((n : \mathbb
N)\) -- number of elements in \(A\) minus 1; \((It : Vec\ A\ (suc\ n))\) -- a
vector containing elements of \(A\) with length \(n + 1\); \((\forall
a\in It)\) -- a proof that any state in \(A\) is also in the vector
\(It\); and \((unique : Unique)\) -- a proof that there is no repeating
elements in \(It\). The vector representation allows us to iterate a
certain set and its subsets, and to extract informaion regarding the
set and a proposition. 

\begin{defn}
\noindent We define \(any\) as a predicate of vector such that it is
true if and only if there exists an element in the vector that satisfies a given
proposition \(P\). 
\end{defn}

\par It is defined in Agda as follow:
\begin{lstlisting}[mathescape=true,xleftmargin=.3\textwidth]
any : {A : Set}{n : $\mathbb N$}(P : A $\to$ Set) $\to$ Vec A n $\to$ Set
any P []        = $\bot$
any P (a :: as) = P a $\uplus$ any P as
\end{lstlisting} 

\begin{lem}
\noindent For a set \(A\) and any proposition \(P\), there exists an
element in \(It\) that satisfies \(P\) if and only if there exists an
element in \(A\) that satisfies \(P\). 
\end{lem}

\begin{proof}
\noindent The proof is quite obvious. Since \(It\) contains all the
elements of \(A\), the statement must be true. However, in Type
Theory, we have to prove it by induction on the vector.
\end{proof}

\begin{defn}
\noindent We define \(all\) as a predicate of vector such that it is
true if and only if all the elements in the vector satisfy a given
proposition \(P\). 
\end{defn}

\par It is defined in Agda as follow:
\begin{lstlisting}[mathescape=true,xleftmargin=.3\textwidth]
all : {A : Set}{n : $\mathbb N$}(P : A $\to$ Set) $\to$ Vec A n $\to$ Set
all P []        = $\top$
all P (a :: as) = P a $\times$ all P as
\end{lstlisting} 

\begin{lem}
\noindent For a set \(A\) and any proposition \(P\), all
the elements in \(It\) that satisfy \(P\) if and only if all the 
elements in \(A\) satisfy \(P\). 
\end{lem}

\begin{proof}
\noindent Again, the proof is quite obvious. Since \(It\) contains all the
elements of \(A\), the statement must be true. However, in Type
Theory, we have to prove it by induction on the vector.
\end{proof}


\section{Languages}
\par The type of languages is defined in \textbf{Language.agda} along with its 
operations and lemmas such as union (\(\cup\)), concatenation
(\(\bullet\)) and closure (\(\star\)). 

\par We represent the set of alphabets \(\Sigma\) as a data type in
Type Theory, i.e. \(\Sigma : Set\). Note that the equality relation of \(\Sigma\) needs to be
decidable. In Agda, they are passed to every module as
parameters, e.g. \(module\ Language (\Sigma : Set)\ (dec : DecEq\
\Sigma)\ where\). 

\begin{defn}
\noindent We first define \(\Sigma^*\) as the set of all
strings over \(\Sigma\). In our approach, it is expressed as a list of
alphabets, i.e. \(\Sigma^* = List\ \Sigma\). 
\end{defn}

\par For example, (\(A :: g :: d :: a :: [\ ]\)) is equivalent to the
string 'Agda' and the empty list \([\ ]\)
represents the empty string (\(\epsilon\)). In this way, the first
alphabet can be extracted from the input string by pattern matching in order to
run a transition from a particular state to another state in an automaton. 

\begin{defn}
\noindent A language is defined as a subset of 
\(\Sigma^*\), i.e. \(Language = Subset\ \Sigma^*\). 
Note that \(Subset\) instead of \(DecSubset\) is used because not
every language is decidable. 
\end{defn}


\subsection{Operations on Languages}

\begin{defn} 
\label{defn:lang_union}
\noindent Suppose \(L_1\) and \(L_2\) are languages, then the union of
the two languages, \(L_1\cup L_2\), is given by the set \(\{w\
|\  w \in L_1\ \vee \ w \in L_2\}\). In Type Theory, we have \(L_1 \cup L_2 = \lambda\ w
\to w \in L_1\ \uplus\ w \in L_2\).
\end{defn}

\begin{defn}
\label{defn:lang_con}
\noindent Suppose \(L_1\) and \(L_2\) are languages, then
the concatenation of the two languages, \(L_1\bullet L_2\), is given
by the set \(\{w\  |\  \exists u\in L_1.\ \exists v\in L_2.\ w = uv\}\). In
Type Theory, we have \(L_1\bullet L_2 = \lambda\ w \to \exists[\
u \in \Sigma^*\ ]\ \exists[\ v \in \Sigma^*\ ] (u \in L_1 \times v \in
L_2 \times w \equiv u\ \Plus\Plus\  v )\).
\end{defn}

\begin{defn}
\label{defn:lang_power}
\noindent Suppose \(L\) is a language, then we define \(L^n\) as
the concatenation of \(L\) with itself over \(n\) times. In Type
Theory, it is defined as a recursive function where \(L^0 = \epsilon\) and
\(L^{k+1}) = L \bullet (L^n)\). 
\end{defn}

\par In Agda, it is defined as follow:
\begin{lstlisting}[mathescape=true,xleftmargin=.3\textwidth]
_$\wedge$_ : Language $\to$ $\mathbb N$ $\to$ Language
L $\wedge$ zero = [ [] ]
L $\wedge$ (suc n) = L $\bullet$ L $\wedge$ n
\end{lstlisting} 


\begin{defn}
\label{defn:lang_star}
\noindent Suppose \(L\) is a language, then the closure of
L, \(L\ast\) is given by the set \(\bigcup_{n \in N} L^n\). In Type
Theory, we have \(L\ \star = \lambda\ w \to \exists [\ n \in \mathbb{N}\
]\ (w \in L \wedge n)\). 
\end{defn}


\section{Regular Expressions and Regular Languages}
\par The types of regular expressions and regular languages are defined in
\textbf{RegularExpression.agda}. 

\begin{defn}
\label{defn:regex}
\noindent Regular expressions over \(\Sigma\) are defined inductively as follow: 
\begin{enumerate}[nolistsep]
  \item \(\O\) is a regular expression denoting the regular language \(\O\);
  \item \(\epsilon\) is a regular expression denoting the regular language \(\{\epsilon\}\);
  \item \(\forall a\in\Sigma.\ a\) is a regular expression denoting the regular language \(\{a\}\);
  \item if \(e_{1}\) and \(e_{2}\) are regular expressions denoting the regular
    languages \(L_1\) and \(L_2\) respectively, then
    \begin{enumerate}[nolistsep]
      \item \(e_{1}\ |\ e_{2}\) is a regular expression denoting the
        regular language \(L_1 \cup L_2\);
      \item \(e_{1}\cdot e_{2}\) is a regular expression denoting the
        regular language \(L_1\bullet L_2\);
      \item \(e_{1}^{\ *}\) is a regular expression denoting the regular
        language \(L_1\ \star\).
     \end{enumerate}
\end{enumerate}
\end{defn}

\par The interpretation of regular expression in Agda is as follow:

\begin{lstlisting}[mathescape=true,xleftmargin=.3\textwidth]
data RegExp : Set where
  $\O$    : RegExp
  $\epsilon$    : RegExp
  $\sigma$    : $\Sigma$ $\to$ RegExp
  _|_ : RegExp $\to$ RegExp $\to$ RegExp
  _$\cdot$_  : RegExp $\to$ RegExp $\to$ RegExp
  _$^*$   : RegExp $\to$ RegExp
\end{lstlisting} 

\par The accepting language of regular expressions is defined as
a function from \(RegExp\) to \(Language\). 

\begin{lstlisting}[mathescape=true,xleftmargin=.3\textwidth]
L$^R$ : RegExp $\to$ Language
L$^R$ $\O$   = $\o$
L$^R$ $\epsilon$   = $[\![\epsilon ]\!]$
L$^R$ ($\sigma$ a) = $[\![\ a\ ]\!]$
L$^R$ (e$_1$ | e$_2$) = L$^R$ e$_1$ $\cup$ L$^R$ e$_2$
L$^R$ (e$_1$ $\cdot$ e$_2$) = L$^R$ e$_1$ $\bullet$ L$^R$ e$_2$
L$^R$ (e$^*$) = (L$^R$ e) $\star$
\end{lstlisting} 